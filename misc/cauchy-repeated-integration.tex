\section*{Cauchy Formula for Repeated Integration}

\subsection*{Basic understanding}

\begin{enumerate}
\item What problem does the Cauchy formula for repeated integration solve?

\item What is meant by the "$n$-th repeated integral" of a function?

\item Write the definition of the $n$-fold repeated integral of a function $f$ with the base point $a$:
  \[
    f^{(-n)}(x) = ?
  \]

\item Under what conditions on $f$ does the formula hold?

\item What is the main result of the Cauchy formula for repeated integration? State it explicitly.
\end{enumerate}

\subsection*{Formula and structure}

\begin{enumerate}
\item Rewrite the repeated integral as a single integral:
  \[
    f^{(-n)}(x) = \frac{1}{(n-1)!} \int_a^x (x - t)^{n-1} f(t)\,dt.
  \]

\item What is the role of the factor $\frac{1}{(n-1)!}$?

\item Why does the kernel $(x - t)^{n-1}$ appear in the formula?

\item What happens in the special case $n = 1$? Verify that the formula reduces correctly.

\item Compute explicitly $f^{(-2)}(x)$ using both:
  \begin{itemize}
    \item the definition (nested integrals),
    \item the Cauchy formula,
  \end{itemize}
  and verify that they agree.
\end{enumerate}

\subsection*{Proof understanding}

\begin{enumerate}
\item What proof technique is used to prove the formula?

\item What is the base case of the induction?

\item Show that for $n=1$:
  \[
    f^{(-1)}(x) = \int_a^x f(t)\,dt.
  \]

\item What assumption is made in the induction step?

\item What must be proven in the induction step?

\item Where is the Leibniz integral rule used in the proof?

\item Compute:
  \[
    \frac{d}{dx} \left[ \frac{1}{n!} \int_a^x (x - t)^n f(t)\,dt \right]
  \]
  and explain why it gives the desired recursive relation.

\item How does the induction step transform an $(n+1)$-fold integral into the desired single-integral form?

\item Why does the induction argument prove the formula for all $n \in \mathbb{N}$?
\end{enumerate}

\subsection*{Conceptual understanding}

\begin{enumerate}
\item Why can repeated integration be "compressed" into a single integral?

\item How is this formula related to convolution?

\item Interpret the kernel $(x - t)^{n-1}$ geometrically or intuitively.

\item How does this formula simplify practical computations?

\item In what situations is this formula especially useful?
\end{enumerate}

\subsection*{Generalizations}

\begin{enumerate}
\item How can the formula be extended to non-integer orders of integration?

\item What replaces the factorial $(n-1)!$ in the generalized version?

\item State the fractional integral version:
  \[
    (J^\alpha f)(x) = ?
  \]

\item What condition must $\alpha$ satisfy?

\item What is the name of this generalization?

\item How does the formula relate to fractional derivatives (when $\alpha < 0$)?
\end{enumerate}

\subsection*{Connections and further ideas}

\begin{enumerate}
\item How is this formula connected to fractional calculus?

\item What is a differintegral?

\item How can repeated integration be used to define differentiation of non-integer order?

\item How is this formula generalized to higher dimensions?

\item What is the Riesz potential?
\end{enumerate}

\subsection*{Practice problems}

\begin{enumerate}
\item Compute:
  \[
    \int_0^x \int_0^{t_1} t_2 \, dt_2 \, dt_1
  \]
  using the Cauchy formula.

\item Compute the $n$-fold integral of $f(t) = 1$.

\item Compute the $n$-fold integral of $f(t) = t^k$.

\item Show that:
  \[
    (J^n f)(x) = (f * g_n)(x)
  \]
  for an appropriate kernel $g_n$.

\item Verify the formula numerically for a simple function (e.g. $f(t)=e^t$).
\end{enumerate}
