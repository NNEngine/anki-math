\section*{Differentiation Under the Integral Sign}

\begin{enumerate}

\item What is the statement of Leibniz's rule for differentiating under the integral sign with constant limits?

\item What is the general form of Leibniz's rule when the limits of integration depend on the parameter?

\item What conditions must be satisfied to justify differentiation under the integral sign (continuity, uniform convergence, dominated convergence, etc.)?

\item What role does continuity of the integrand and its partial derivative play in applying differentiation under the integral sign?

\item How does the Dominated Convergence Theorem justify differentiation under the integral sign?

\item When can uniform convergence be used to justify differentiation under the integral sign?

\item What is the difference between pointwise convergence and uniform convergence in this context?

\item How do you compute $\frac{d}{d\alpha} \int_a^b f(x,\alpha),dx$?

\item How do you compute $\frac{d}{d\alpha} \int_{a(\alpha)}^{b(\alpha)} f(x,\alpha),dx$?

\item What additional terms appear when the limits of integration depend on the parameter?

\item How do you apply differentiation under the integral sign to evaluate difficult integrals?

\item What is the idea behind Feynman's technique for evaluating integrals using parameters?

\item How can introducing a parameter simplify the evaluation of an integral?

\item Give an example where differentiating under the integral sign transforms an integral into a simpler one.

\item How do you recover the original integral after differentiating with respect to a parameter?

\item What initial or boundary conditions are needed after integrating back with respect to the parameter?

\item How do you handle constants of integration when using this method?

\item When is it valid to interchange differentiation and integration?

\item What are common pitfalls when applying differentiation under the integral sign?

\item How does this technique apply to improper integrals?

\item What extra care is needed when the interval of integration is infinite?

\item How do you justify differentiation under the integral sign for $\int_0^\infty f(x,\alpha),dx$?

\item Can differentiation under the integral sign be applied multiple times? Under what conditions?

\item What is the relationship between differentiation under the integral sign and parameter-dependent integrals?

\item How does this method connect to solving differential equations?

\item How can this technique be used in probability theory (e.g., moment generating functions)?

\item How does differentiation under the integral sign relate to Fourier transforms or Laplace transforms?

\item What are some classic integrals that are evaluated using this method?

\item How would you construct your own parameterized integral to apply this technique?

\item How do you verify that your final answer is correct after applying the method?

\end{enumerate}
