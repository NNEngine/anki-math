\section*{Hessian Matrix}

\subsection*{Basic Definitions}
\begin{enumerate}
\item What is the Hessian matrix of a scalar-valued function $f:\mathbb{R}^n \to \mathbb{R}$?
\item Write the general form of the Hessian matrix for a function $f(x_1, x_2, \dots, x_n)$.
\item What are the entries of the Hessian matrix in terms of partial derivatives?
\item For $f(x,y)$, write the Hessian explicitly.
\end{enumerate}

\subsection*{Computation}
\begin{enumerate}
\item Compute the Hessian of $f(x,y) = x^2 + y^2$.
\item Compute the Hessian of $f(x,y) = x^2 y + y^3$.
\item Compute the Hessian of $f(x,y,z) = xyz$.
\item Compute the Hessian of $f(x,y) = e^{x^2 + y^2}$.
\end{enumerate}

\subsection*{Interpretation}
\begin{enumerate}
\item What information does the Hessian matrix encode about a function?
\item How is the Hessian related to the second-order Taylor expansion?
\item What does the Hessian tell you about local curvature?
\end{enumerate}

\subsection*{Critical Points and Optimization}
\begin{enumerate}
\item What is a critical point of a multivariable function?
\item State the second derivative test using the Hessian for functions of two variables.
\item What are the conditions on the Hessian for:
\item a local minimum?
\item a local maximum?
\item a saddle point?
\item What does it mean for the Hessian to be positive definite?
\item What does it mean for the Hessian to be negative definite?
\item What happens if the Hessian is indefinite?
\item What happens if the determinant of the Hessian is zero?
\end{enumerate}

\subsection*{Matrix Properties}
\begin{enumerate}
\item When is the Hessian symmetric?
\item State the conditions under which mixed partial derivatives are equal.
\item Why is symmetry of the Hessian important?
\end{enumerate}

\subsection*{Eigenvalues and Geometry}
\begin{enumerate}
\item How are eigenvalues of the Hessian related to curvature?
\item What does it mean if all eigenvalues are positive?
\item What does it mean if eigenvalues have mixed signs?
\item How do eigenvectors of the Hessian relate to principal directions?
\end{enumerate}

\subsection*{Advanced Connections}
\begin{enumerate}
\item How does the Hessian relate to convexity of a function?
\item State the condition for a function to be convex using the Hessian.
\item What is the Hessian in the context of optimization algorithms (e.g., Newton's method)?
\item How is the Hessian used in quadratic approximation?
\item What is the bordered Hessian and when is it used?
\end{enumerate}

\subsection*{Conceptual Checks}
\begin{enumerate}
\item Why does the first derivative test fail for multivariable functions without second-order information?
\item Can a function have a zero gradient but not be an extremum? Explain using the Hessian.
\item How does the Hessian generalize the second derivative from single-variable calculus?
\end{enumerate}
