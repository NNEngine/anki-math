\documentclass{article}
\usepackage[utf8]{inputenc}
\usepackage[a4paper, margin=1in]{geometry}
\usepackage{hyperref}
\usepackage{amsmath, amssymb, amsthm}

\hypersetup{
    pdftitle={Active Recall Questions},
    pdfauthor={Mathematics Study Notes},
    pdfsubject={Mathematics},
    pdfkeywords={Mathematics, Active Recall, Self-Check Questions},
    pdfcreator={LaTeX},
    pdfproducer={pdfLaTeX}
}

\hypersetup{
    colorlinks=true,
    linkcolor=blue,
    urlcolor=blue
}

\urlstyle{same}

\title{Active Recall Questions}
\author{Mathematics Study Notes}
\date{\today}

\begin{document}

\maketitle

\section*{Introduction}

Mathematics is not a subject that can be mastered through passive reading alone. True understanding requires the ability to retrieve concepts, definitions, and problem-solving strategies from memory. One of the most effective techniques for achieving this is \emph{active recall}, especially when implemented through a structured list of self-check questions.

\section*{What is Active Recall?}

Active recall is the process of deliberately retrieving information from memory rather than re-exposing oneself to it. Instead of rereading notes or textbooks, the learner attempts to answer questions, reconstruct proofs, or solve problems without looking at the solution.

In mathematics, this might include:
\begin{itemize}
    \item Stating definitions (e.g., convergence of a series),
    \item Recalling theorems (e.g., the Integral Test),
    \item Re-deriving formulas,
    \item Solving representative problems from memory.
\end{itemize}

\section*{Role of Self-Check Questions}

A list of self-check questions provides structure to active recall. These questions act as prompts that guide the learner through the essential components of a topic. For example, in studying infinite series, one might ask:
\begin{itemize}
    \item What is the definition of convergence?
    \item When can the Integral Test be applied?
    \item Why does removing finitely many terms not affect convergence?
\end{itemize}

Such questions ensure that the learner engages with both conceptual understanding and procedural knowledge.

\section*{Why It Works}

Active recall strengthens memory through effortful retrieval. Each time information is successfully recalled, the neural pathways associated with that knowledge become more robust. This leads to:
\begin{itemize}
    \item Improved long-term retention,
    \item Faster recall during problem solving,
    \item Better ability to connect different concepts.
\end{itemize}

Moreover, attempting to answer questions reveals gaps in understanding. These gaps are often invisible during passive review but become immediately apparent during recall attempts.

\clearpage

\section*{4. Introduction to Differential Equations}

\url{https://openstax.org/books/calculus-volume-2/pages/4-introduction}

\subsection*{4.1 Basics of Differential Equations}

\url{https://openstax.org/books/calculus-volume-2/pages/4-1-basics-of-differential-equations}

\begin{enumerate}

\item What is a differential equation?

\item What is meant by a solution to a differential equation?

\item How can you verify that a function $y=f(x)$ is a solution to a given differential equation?

\item Give an example of a differential equation and one of its solutions.

\item Why are solutions to differential equations often not unique?

\item What property of derivatives explains the non-uniqueness of solutions?

\item What is the order of a differential equation?

\item How do you determine the order of a differential equation from its expression?

\item What is the order of the differential equation $y' - 4y = x^2 - 3x + 4$?

\item What is the order of the differential equation $x^2 y''' - 3x y'' + x y' - 3y = \sin x$?

\item What is a general solution of a differential equation?

\item Why does a general solution typically include an arbitrary constant $C$?

\item For the differential equation $y' = 2x$, what is its general solution?

\item What does the family of functions $y = x^2 + C$ represent?

\item What is a particular solution of a differential equation?

\item How do you obtain a particular solution from a general solution?

\item What is an initial-value problem?

\item How does an initial condition help determine a unique solution?

\item Solve the initial-value problem:
  \[
    y' = 2x, \quad y(2) = 7.
  \]

\item What is the difference between a general solution and a particular solution?

\item What does it mean for a function to satisfy a differential equation?

\item Verify that $y = e^{-3x} + 2x + 3$ is a solution of
  \[
    y' + 3y = 6x + 11.
  \]

\item Verify that $y = 2e^{3x} - 2x - 2$ is a solution of
  \[
    y' - 3y = 6x + 4.
  \]

\item When checking a solution, why must both $y$ and its derivatives be substituted into the equation?

\item Can two different functions be solutions to the same differential equation? Explain why.

\item What role does the constant $C$ play geometrically in the family of solutions?

\item How would the graph of $y = x^2 + C$ change as $C$ varies?

\item Why is identifying the order of a differential equation useful?

\item What information is needed in addition to a differential equation to determine a unique solution?

\end{enumerate}
\clearpage
\subsection*{4.2 Direction Fields and Numerical Methods}

\url{https://openstax.org/books/calculus-volume-2/pages/4-2-direction-fields-and-numerical-methods}

\begin{enumerate}

\item What is a direction field (slope field) for a differential equation?

\item For what type of differential equations are direction fields typically used?

\item What is the general form of a first-order differential equation used in direction fields?

\item What does each small line segment in a direction field represent?

\item How is the slope of a line segment at a point $(x,y)$ determined?

\item How do you construct a direction field for a differential equation $y' = f(x,y)$?

\item Why can a direction field be drawn without solving the differential equation explicitly?

\item What qualitative information can a direction field provide about solutions?

\item What is a solution curve in the context of a direction field?

\item How can you sketch a solution curve using a direction field?

\item Why do solution curves follow the line segments in a direction field?

\item What is an initial condition in the context of direction fields?

\item How do you use an initial point $(x_0,y_0)$ to sketch a particular solution?

\item Why do solution curves corresponding to different initial conditions typically not intersect?

\item What is an equilibrium solution?

\item How can equilibrium solutions be identified from a differential equation?

\item How do equilibrium solutions appear in a direction field?

\item What does it mean for an equilibrium solution to be stable?

\item What does it mean for an equilibrium solution to be unstable?

\item How can you determine stability visually from a direction field?

\item What is the purpose of numerical methods in differential equations?

\item Why are numerical methods needed when solving differential equations?

\item What is Euler’s Method?

\item What type of problems is Euler’s Method used to approximate?

\item What is the idea behind Euler’s Method in terms of tangent line approximation?

\item What is the step size $h$ in Euler’s Method?

\item How does the choice of step size affect the accuracy of Euler’s Method?

\item What is the iterative formula for Euler’s Method?

\item Starting from $(x_n, y_n)$, how do you compute $(x_{n+1}, y_{n+1})$?

\item Write the update rule:
  \[
    y_{n+1} = y_n + h f(x_n, y_n).
  \]

\item How is $x_{n+1}$ computed from $x_n$?

\item Apply Euler’s Method to approximate the solution of
  \[
    y' = f(x,y), \quad y(x_0)=y_0
  \]
  for one step.

\item Apply Euler’s Method for two steps and express the result explicitly.

\item What geometric idea explains why Euler’s Method works?

\item Why does Euler’s Method accumulate error over multiple steps?

\item What is the difference between the exact solution and a numerical approximation?

\item How can decreasing the step size $h$ improve the approximation?

\item What is the trade-off when choosing a very small step size?

\item How is Euler’s Method related to linear approximation (tangent lines)?

\item In what sense does Euler’s Method “follow” the direction field?

\item How would you approximate a solution curve using only a direction field (without formulas)?

\item What are the limitations of direction fields?

\item What are the limitations of Euler’s Method?

\item When would you prefer a graphical method over an analytical solution?

\item When would you prefer a numerical method over solving analytically?

\item How can direction fields and Euler’s Method be used together?

\end{enumerate}
\clearpage
\subsection*{4.3 Separable Equations}

\url{https://openstax.org/books/calculus-volume-2/pages/4-3-separable-equations}

\begin{enumerate}

\item What is a separable differential equation?

\item In what form must a differential equation be written to be considered separable?

\item Write the general form:
  \[
    \frac{dy}{dx} = f(x)g(y).
  \]

\item Why is such an equation called “separable”?

\item What does it mean to separate variables in a differential equation?

\item How do you rewrite a separable differential equation in differential form?

\item Write the separated form:
  \[
    \frac{dy}{g(y)} = f(x)\,dx.
  \]

\item What is the main idea behind the method of separation of variables?

\item What are the steps of the separation of variables method?

\item Why must you check for values where $g(y)=0$ before separating variables?

\item What do solutions of $g(y)=0$ represent?

\item Why can constant solutions be lost during the separation process?

\item After separating variables, what is the next step?

\item Why do we integrate both sides of the equation?

\item What form does the equation take after integration?

\item Why is an arbitrary constant $C$ introduced after integration?

\item When solving, why might it not be possible to express $y$ explicitly as a function of $x$?

\item What is an implicit solution?

\item Give an example of an implicit solution arising from separation of variables.

\item How do you apply an initial condition to a separable differential equation?

\item What is an initial-value problem in this context?

\item After integrating, how do you solve for the constant using an initial condition?

\item Solve the general structure:
  \[
    \int \frac{1}{g(y)}\,dy = \int f(x)\,dx.
  \]

\item For the equation $y' = (x^2 - 4)(3y+2)$, how do you separate variables?

\item What constant solution arises from $3y+2=0$?

\item After separation, what integral must be computed:
  \[
    \int \frac{dy}{3y+2}?
  \]

\item What substitution is useful for integrating expressions like $\frac{1}{3y+2}$?

\item Why is substitution often needed when integrating the $y$-side?

\item How do you interpret the final solution after integrating both sides?

\item For the equation $y' = (2x+3)(y^2 - 4)$, how do you separate variables?

\item What constant solutions arise from $y^2 - 4 = 0$?

\item Why do these constant solutions need to be included separately?

\item What integration technique is typically used for expressions like $\frac{1}{y^2 - 4}$?

\item How does partial fraction decomposition appear in separable equations?

\item Why are integration techniques from earlier calculus essential here?

\item What is an autonomous differential equation?

\item How can you recognize an autonomous equation from its form?

\item Why is an equation of the form $y' = g(y)$ separable?

\item What is the general solution to an equation of the form $y' = f(x)$?

\item What is the general solution to an equation of the form $y' = g(y)$?

\item How does separation of variables generalize these simpler cases?

\item In what situations is separation of variables especially useful?

\item Why are separable equations common in physics and engineering?

\item What are the limitations of the separation of variables method?

\item Can every differential equation be solved using separation of variables?

\item How can you recognize when a differential equation is not separable?

\item What is the overall strategy when solving separable differential equations?

\end{enumerate}
\clearpage
\subsection*{4.4 The Logistic Equation}

\url{https://openstax.org/books/calculus-volume-2/pages/4-4-the-logistic-equation}

\begin{enumerate}

\item What real-world limitation motivates the logistic model of population growth?

\item Why is exponential growth not realistic for large populations?

\item What is the carrying capacity of a population?

\item What does the symbol $K$ represent in the logistic equation?

\item What does the symbol $r$ represent in the logistic equation?

\item Write the logistic differential equation:
  \[
    \frac{dP}{dt} = rP\left(1 - \frac{P}{K}\right).
  \]

\item Why is the logistic equation nonlinear?

\item What happens to $\frac{dP}{dt}$ when $P$ is very small?

\item How does the logistic equation behave when $P \ll K$?

\item How does the logistic model approximate exponential growth for small populations?

\item What happens to $\frac{dP}{dt}$ when $P = K$?

\item What does it mean physically when $P = K$?

\item What happens to $\frac{dP}{dt}$ when $P > K$?

\item Why does the population decrease when $P > K$?

\item What are the equilibrium solutions of the logistic equation?

\item Solve:
  \[
    rP\left(1 - \frac{P}{K}\right) = 0.
  \]

\item Why are $P=0$ and $P=K$ equilibrium solutions?

\item What is meant by stability of an equilibrium solution?

\item Is $P=0$ stable, unstable, or semi-stable? Explain.

\item Is $P=K$ stable, unstable, or semi-stable? Explain.

\item What is a phase line?

\item How do you construct a phase line for the logistic equation?

\item What does the phase line indicate about the sign of $\frac{dP}{dt}$?

\item For which values of $P$ is the population increasing?

\item For which values of $P$ is the population decreasing?

\item How does the phase line describe long-term behavior of solutions?

\item What happens to $P(t)$ as $t \to \infty$ when $0 < P_0 < K$?

\item What happens to $P(t)$ as $t \to \infty$ when $P_0 > K$?

\item Why does the solution approach $K$ but never reach it exactly?

\item What is the general strategy to solve the logistic differential equation?

\item Why is the logistic equation separable?

\item Rewrite the logistic equation in separated form.

\item What algebraic manipulation is needed before integrating?

\item Why is partial fraction decomposition used in solving the logistic equation?

\item What is the general solution of the logistic differential equation?

\item Write the solution:
  \[
    P(t)=\frac{P_0 K e^{rt}}{(K-P_0)+P_0 e^{rt}}.
  \]

\item What is the role of the initial condition $P(0)=P_0$?

\item How does the solution behave as $t \to \infty$?

\item How does the solution behave as $t \to -\infty$?

\item What is the shape of the graph of a logistic function?

\item Why is the logistic curve called sigmoidal (S-shaped)?

\item What is a point of inflection?

\item Why does the logistic solution have a point of inflection?

\item At what population value does the inflection point occur?

\item Why is the growth rate maximal at $P = \frac{K}{2}$?

\item How can you find the inflection point using the second derivative?

\item What happens to concavity before and after the inflection point?

\item How does the logistic model improve on exponential growth models?

\item In what types of real-world systems is the logistic equation used?

\item What assumptions underlie the logistic model?

\item What are the limitations of the logistic equation?

\end{enumerate}
\clearpage
\subsection*{4.5 First-order Linear Equations}

\url{https://openstax.org/books/calculus-volume-2/pages/4-5-first-order-linear-equations}

\begin{enumerate}

\item What is a first-order linear differential equation?

\item What is the general (standard) form of a first-order linear differential equation?

\item Write the standard form:
  \[
    y' + p(x)y = q(x).
  \]

\item How can any first-order linear differential equation be rewritten into standard form?

\item What roles do the functions $p(x)$ and $q(x)$ play?

\item Why is the equation called ``linear''?

\item What distinguishes a linear differential equation from a separable one?

\item Can every linear differential equation be solved by separation of variables? Why or why not?

\item What is an integrating factor?

\item Why is an integrating factor useful when solving linear differential equations?

\item What property do we want the integrating factor to create?

\item What is the formula for the integrating factor $\mu(x)$?

\item Write:
  \[
    \mu(x) = e^{\int p(x)\,dx}.
  \]

\item How is the integrating factor derived?

\item What differential equation does $\mu(x)$ satisfy?

\item Write:
  \[
    \mu'(x) = \mu(x)p(x).
  \]

\item After multiplying by $\mu(x)$, what form does the equation take?

\item Why does the left-hand side become a derivative of a product?

\item Write the identity:
  \[
    \frac{d}{dx}(\mu(x)y) = \mu(x)y' + \mu(x)p(x)y.
  \]

\item After multiplying by $\mu(x)$, what equation do we obtain?

\item Write:
  \[
    \frac{d}{dx}(\mu(x)y) = \mu(x)q(x).
  \]

\item What is the next step after obtaining this form?

\item Why can both sides now be integrated easily?

\item After integrating, what form does the solution take?

\item Write the general solution:
  \[
    \mu(x)y = \int \mu(x)q(x)\,dx + C.
  \]

\item How do you solve for $y(x)$ after integration?

\item What is the full step-by-step strategy for solving first-order linear equations?

\item List the five steps of the method.

\item Why is it important to first rewrite the equation in standard form?

\item What happens if the coefficient of $y'$ is not 1?

\item How do you handle an equation of the form $a(x)y' + b(x)y = c(x)$?

\item Why is dividing by $a(x)$ necessary?

\item What is an initial-value problem in this context?

\item How do you apply an initial condition to determine the constant $C$?

\item Solve symbolically:
  \[
    y' + p(x)y = 0.
  \]

\item What is the general solution to the homogeneous equation?

\item How does the homogeneous solution relate to the general solution?

\item What is the difference between homogeneous and nonhomogeneous equations?

\item What happens when $q(x)=0$?

\item What happens when $q(x)\neq 0$?

\item Why does the solution involve an integral of $\mu(x)q(x)$?

\item What integration techniques are often required in solving these equations?

\item How does the integrating factor method simplify the problem conceptually?

\item What is the geometric interpretation of multiplying by $\mu(x)$?

\item Why does the method always work for first-order linear equations?

\item What are common mistakes when applying the integrating factor method?

\item How can you verify that a solution is correct?

\item In what applications do first-order linear equations appear?

\item How are they used in modeling motion with air resistance?

\item How are they used in electrical circuits?

\item Why are first-order linear equations important in applied mathematics?

\end{enumerate}
\clearpage

\section*{5. Sequences and Series}

\url{https://openstax.org/books/calculus-volume-2/pages/5-introduction}

\subsection*{5.1 Sequences}

\url{https://openstax.org/books/calculus-volume-2/pages/5-1-sequences}

\begin{enumerate}

\item What is an infinite sequence? How is it typically denoted?

\item How can a sequence be interpreted as a function? What is its domain?

\item What is the index of a sequence, and what role does it play?

\item What is an explicit formula for a sequence? Give an example.

\item What is a recursive definition of a sequence? How does it differ from an explicit formula?

\item Given a recursive sequence, how can you compute its first few terms?

\item What does it mean to graph a sequence? How does it differ from graphing a function?

\item What is the limit of a sequence?

\item What does it mean for a sequence ${a_n}$ to converge to a number $L$?

\item Write the formal $\epsilon$–$N$ definition of convergence of a sequence.

\item What does it mean for a sequence to diverge?

\item What are some different ways a sequence can diverge?

\item What does it mean for a sequence to diverge to infinity? To negative infinity?

\item Why does changing a finite number of terms of a sequence not affect its convergence?

\item How can limits of functions be used to determine limits of sequences?

\item State the theorem relating $\lim_{x\to\infty} f(x)$ and $\lim_{n\to\infty} a_n$ when $a_n = f(n)$.

\item Evaluate $\lim_{n\to\infty} \frac{1}{n}$ using a function-based argument.

\item What happens to $r^n$ as $n \to \infty$ when:
  \begin{itemize}
    \item $0 < r < 1$?
    \item $r = 1$?
    \item $r > 1$?
  \end{itemize}

\item How can limit laws be applied to sequences?

\item How do you find the limit of a sequence that is a sum or product of simpler sequences?

\item What is a bounded sequence?

\item What does it mean for a sequence to be increasing? Decreasing?

\item What is a monotonic (monotone) sequence?

\item State the Monotone Convergence Theorem.

\item Why is the Monotone Convergence Theorem useful?

\item Give an example of a sequence that is bounded and increasing.

\item Give an example of a sequence that is unbounded.

\item Give an example of a sequence that oscillates and does not converge.

\item How can you determine convergence from the behavior of the terms as $n \to \infty$?

\item What is the difference between intuition (“approaches a value”) and the formal definition of convergence?

\item How can sequences be used to model real-world processes or iterative procedures?

\item How are sequences related to infinite series?

\end{enumerate}
\clearpage
\subsection*{5.2 Infinite Series}

\url{https://openstax.org/books/calculus-volume-2/pages/5-2-infinite-series}

\begin{enumerate}

\item What is an infinite series? How is it related to a sequence?

\item How is an infinite series written using summation notation?

\item What is the difference between a sequence ${a_n}$ and the series $\sum_{n=1}^{\infty} a_n$?

\item What is a partial sum of a series? How is the $k$-th partial sum $S_k$ defined?

\item How can an infinite series be defined as a limit of partial sums?

\item What does it mean for an infinite series to converge?

\item What does it mean for an infinite series to diverge?

\item What is the relationship between the convergence of a series and the convergence of its sequence of partial sums?

\item What does it mean for a series to diverge to infinity?

\item How can you determine whether a series converges by examining its partial sums?

\item What is a geometric series? Write its general form.

\item What is the common ratio of a geometric series?

\item Under what condition does a geometric series converge?

\item What is the formula for the sum of a convergent geometric series?

\item What happens if the common ratio $r$ of a geometric series satisfies $|r| \geq 1$?

\item How can you derive the formula for the sum of a geometric series?

\item What is a telescoping series?

\item How do terms cancel in a telescoping series?

\item How can you evaluate the sum of a telescoping series?

\item Why are telescoping series often easier to evaluate than other series?

\item Give an example of a telescoping series and describe its behavior.

\item What is the harmonic series? Write its general form.

\item Does the harmonic series converge or diverge?

\item Why is it not sufficient for $a_n \to 0$ to guarantee that $\sum a_n$ converges?

\item What is the divergence test (nth-term test for divergence)?

\item If $\lim_{n \to \infty} a_n \neq 0$, what can you conclude about $\sum a_n$?

\item If $\lim_{n \to \infty} a_n = 0$, what can you conclude about $\sum a_n$?

\item How are infinite series used in applications such as approximations or modeling?

\item How does the concept of an infinite series extend the idea of a finite sum?

\item How can infinite series be used to represent functions?

\item What is the intuitive meaning of “adding infinitely many terms”?

\item How does the order of terms affect the value of a finite sum? Does this idea carry over to infinite series?

\end{enumerate}
\clearpage
\subsection*{5.3 The Divergence and Integral Tests}

\url{https://openstax.org/books/calculus-volume-2/pages/5-3-the-divergence-and-integral-tests}

\begin{enumerate}

\item Why is it often difficult to determine convergence of a series by directly evaluating partial sums?

\item What is the divergence test (nth-term test)?

\item State the divergence test formally.

\item Why does $\sum_{n=1}^{\infty} a_n$ converging imply that $a_n \to 0$?

\item If $\lim_{n\to\infty} a_n \neq 0$, what can you conclude about the series $\sum a_n$?

\item If $\lim_{n\to\infty} a_n = 0$, what can you conclude about the series $\sum a_n$?

\item Why is the divergence test only useful for proving divergence and not convergence?

\item Give an example of a series where $a_n \to 0$ but the series diverges.

\item How do you apply the divergence test in practice to a given series?

\item What kinds of limits of $a_n$ guarantee divergence (nonzero limit, limit does not exist)?

\item What is the idea behind comparing a series to an improper integral?

\item What is the integral test?

\item What conditions must a function $f(x)$ satisfy to apply the integral test (positivity, continuity, monotonicity)?

\item How are the series $\sum_{n=1}^{\infty} a_n$ and the integral $\int_1^\infty f(x),dx$ related in the integral test?

\item State the integral test formally.

\item If the improper integral $\int_1^\infty f(x),dx$ converges, what can you conclude about the series?

\item If the improper integral diverges, what can you conclude about the series?

\item How can the integral test be used to prove that the harmonic series diverges?

\item What is a $p$-series? Write its general form.

\item For what values of $p$ does the $p$-series $\sum_{n=1}^{\infty} \frac{1}{n^p}$ converge?

\item For what values of $p$ does the $p$-series diverge?

\item How can the integral test be used to establish the convergence properties of $p$-series?

\item What is the remainder $R_N$ of a series after $N$ terms?

\item How can the integral test be used to estimate the remainder of a series?

\item State the inequality involving the remainder $R_N$ and integrals.

\item How do upper and lower bounds for $R_N$ arise from the integral test?

\item Why is it useful to estimate the error when approximating a series by partial sums?

\item How can you choose $N$ to ensure the error is less than a desired tolerance?

\item What is the geometric interpretation of the integral test using areas under curves?

\item How does monotonicity of $f(x)$ play a role in bounding the series with integrals?

\item In what situations is the integral test particularly useful compared to other tests?

\end{enumerate}
\clearpage
\subsection*{5.4 Comparison Tests}

\url{https://openstax.org/books/calculus-volume-2/pages/5-4-comparison-tests}

\begin{enumerate}

\item What is the main idea behind comparison tests for infinite series?

\item For what type of series (in terms of sign of terms) are comparison tests typically used?

\item Why are geometric series and $p$-series commonly used in comparison tests?

\item State the Direct Comparison Test for series $\sum a_n$ and $\sum b_n$ with positive terms.

\item If $0 \le a_n \le b_n$ and $\sum b_n$ converges, what can be concluded about $\sum a_n$?

\item If $0 \le b_n \le a_n$ and $\sum b_n$ diverges, what can be concluded about $\sum a_n$?

\item In which cases is the Direct Comparison Test inconclusive?

\item How do you choose an appropriate comparison series $b_n$ for a given series $a_n$?

\item Why is it useful to compare a series to a $p$-series of the form $\sum \frac{1}{n^p}$?

\item What is the behavior of the $p$-series $\sum \frac{1}{n^p}$ for different values of $p$?

\item What is the behavior of a geometric series $\sum ar^n$ depending on the value of $r$?

\item State the Limit Comparison Test for two series $\sum a_n$ and $\sum b_n$ with positive terms.

\item What does it mean if
  \[
    \lim_{n \to \infty} \frac{a_n}{b_n} = c,
  \]
  where $0 < c < \infty$?

\item What can be concluded if the limit in the Limit Comparison Test is $0$?

\item What can be concluded if the limit in the Limit Comparison Test is $\infty$?

\item When is the Limit Comparison Test more useful than the Direct Comparison Test?

\item Why does the Limit Comparison Test work even when inequalities between $a_n$ and $b_n$ are hard to establish?

\item What are common functions or expressions you simplify when applying the Limit Comparison Test?

\item How do you simplify rational expressions in $n$ when applying comparison tests?

\item How do logarithmic terms (e.g., $\ln n$) affect comparison with $p$-series?

\item How do roots (e.g., $\sqrt{n}$) affect comparison with $p$-series?

\item Give an example of a series that can be compared to $\sum \frac{1}{n^2}$ and explain why.

\item Give an example of a series that can be compared to $\sum \frac{1}{n}$ and explain why.

\item What is the general strategy for proving convergence using comparison tests?

\item What is the general strategy for proving divergence using comparison tests?

\item Why must terms $a_n$ and $b_n$ be positive for these tests?

\item Can comparison tests determine the exact sum of a series? Why or why not?

\item How do comparison tests relate to the Integral Test conceptually?

\item What role do asymptotic behaviors of functions play in the Limit Comparison Test?

\item How can you justify that two sequences have the same “growth rate” for comparison?

\item When comparing $\frac{1}{n^2+1}$ to $\frac{1}{n^2}$, why does the comparison work?

\end{enumerate}
\clearpage
\subsection*{5.5 Alternating Series}

\url{https://openstax.org/books/calculus-volume-2/pages/5-5-alternating-series}

\begin{enumerate}

\item What is an alternating series? Write its general form.

\item How can an alternating series be written using the factor $(-1)^n$ or $(-1)^{n+1}$?

\item In the representation $\sum_{n=1}^{\infty} (-1)^{n+1} b_n$, what conditions are imposed on $b_n$?

\item State the Alternating Series Test (Leibniz Test).

\item What two conditions must the sequence $\{b_n\}$ satisfy for the Alternating Series Test to apply?

\item Why is the condition $\lim_{n \to \infty} b_n = 0$ necessary for convergence?

\item Does the Alternating Series Test guarantee absolute convergence or only convergence? Explain.

\item Give an example of a series that converges by the Alternating Series Test.

\item What is meant by absolute convergence?

\item What is meant by conditional convergence?

\item What is the relationship between absolute convergence and convergence of a series?

\item Give an example of a series that converges conditionally but not absolutely.

\item How can you test whether an alternating series converges absolutely?

\item If $\sum |a_n|$ diverges but $\sum a_n$ converges, what type of convergence does the series have?

\item What is the alternating harmonic series? Does it converge absolutely or conditionally?

\item Define the nth partial sum $S_N$ of a series.

\item What is the remainder $R_N$ of a series?

\item For an alternating series satisfying the Alternating Series Test, what inequality bounds the remainder $R_N$?

\item Write the inequality relating $|R_N|$ and $b_{N+1}$.

\item What does the remainder estimate tell us about the error when approximating a sum by $S_N$?

\item How can you use the remainder estimate to determine how many terms are needed for a desired accuracy?

\item If you approximate the sum of an alternating series by $S_N$, what is the maximum possible error?

\item Why do partial sums of an alternating series tend to oscillate around the true sum?

\item Describe the behavior of even and odd partial sums in an alternating series that satisfies the test.

\item How can you estimate the sum of an alternating series to within a given tolerance $\varepsilon$?

\item For the series $\sum_{n=1}^{\infty} \frac{(-1)^{n+1}}{n^2}$, how would you bound the error after N terms?

\item What role does monotonicity $(b_{n+1} \le b_n)$ play in the Alternating Series Test?

\item Can an alternating series converge if the sequence $b_n$ is not decreasing? What does the test say about this?

\item Compare the convergence behavior of $\sum \frac{(-1)^{n+1}}{n}$ and $\sum \frac{1}{n}$.

\item Why is it often difficult to compute the exact sum of an alternating series?

\item What practical advantage does the Alternating Series Estimation Theorem provide?

\item How does the Alternating Series Test differ from comparison or ratio tests?

\item When analyzing a series, why is it useful to check absolute convergence first?

\item What happens if $\lim_{n \to \infty} a_n \neq 0$ for an alternating series?

\item How would you structure a complete convergence test for a series that alternates in sign?

\end{enumerate}
\clearpage
\subsection*{5.6 Ratio and Root Tests}

\url{https://openstax.org/books/calculus-volume-2/pages/5-6-ratio-and-root-tests}

\begin{enumerate}

\item Why is the condition $\lim_{n\to\infty} a_n = 0$ not sufficient to guarantee convergence of $\sum a_n$?

\item Give an example of two series where $a_n \to 0$ in both cases, but one converges and the other diverges.

\item What does it mean for the terms of a series to approach zero “fast enough”?

\item State the Ratio Test for a series $\sum a_n$.

\item Define
  \[
    \rho = \lim_{n\to\infty} \left| \frac{a_{n+1}}{a_n} \right|.
  \]

\item What conclusion can be drawn if $\rho < 1$ in the Ratio Test?

\item What conclusion can be drawn if $\rho > 1$ (or $\rho = \infty$) in the Ratio Test?

\item What happens if $\rho = 1$ in the Ratio Test?

\item What type of convergence does the Ratio Test establish when $\rho < 1$?

\item For what kinds of series is the Ratio Test especially useful?

\item Apply the Ratio Test conceptually: what happens to the ratio $\left|\frac{a_{n+1}}{a_n}\right|$ for factorial-type expressions?

\item State the Root Test for a series $\sum a_n$.

\item Define
  \[
    \rho = \lim_{n\to\infty} \sqrt[n]{|a_n|}.
  \]

\item What conclusion can be drawn if $\rho < 1$ in the Root Test?

\item What conclusion can be drawn if $\rho > 1$ in the Root Test?

\item What happens if $\rho = 1$ in the Root Test?

\item Compare the Ratio Test and Root Test: what is similar about their conclusions?

\item In what situations is the Root Test particularly useful?

\item How does the Root Test behave for expressions involving powers like $(b_n)^n$?

\item What does it mean for a test to be “inconclusive”?

\item Give an example of a series where the Ratio Test is inconclusive.

\item Give an example of a series where the Root Test is inconclusive.

\item Why might one test succeed when another test fails?

\item What is absolute convergence, and how do these tests relate to it?

\item Why do both the Ratio and Root Tests use absolute values?

\item How are these tests useful for power series?

\item What general strategy can be used to choose an appropriate convergence test for a given series?

\item When should you consider using the Ratio Test over other tests?

\item When should you consider using the Root Test over other tests?

\item What is a good first step before applying either the Ratio or Root Test?

\item How can simplifying expressions help when applying these tests?

\item Why are these tests often easier to apply than comparison tests?

\item Can the Ratio or Root Test determine conditional convergence?

\item How do these tests relate to geometric series?

\item How can you sometimes modify a series to make the Root Test easier to apply?

\item Why is it important to recognize patterns like factorials, exponentials, or powers before choosing a test?

\item Summarize the decision process after computing $\rho$ in either test.

\end{enumerate}
\clearpage

\section*{6. Power Series}

\url{https://openstax.org/books/calculus-volume-2/pages/6-introduction}

\subsection*{6.1 Power Series and Functions}

\url{https://openstax.org/books/calculus-volume-2/pages/6-1-power-series-and-functions}

\begin{enumerate}

\item What is the general form of a power series centered at \(x=0\)?

\item Write the general form of a power series centered at \(x=a\).

\item Explain in your own words why a power series can be thought of as an “infinite polynomial.”

\item Give two examples of power series centered at \(x=0\) with explicit formulas for their coefficients \(c_n\).

\item What does it mean for a power series to be “centered at” a specific point?

\item State the three possible convergence behaviors of a power series centered at \(x=a\).

\item Define the \emph{interval of convergence} for a power series.

\item Define the \emph{radius of convergence} of a power series.

\item Why does every power series always converge at its center?

\item Using the geometric series
  \[
    \sum_{n=0}^\infty x^n,
  \]
  determine its interval and radius of convergence.

\item How can the Ratio Test be used to help find the radius of convergence of a power series?

\item True/False: If a power series converges for some value \(x_0\), then it converges for all \(x\) with \(|x-a|<|x_0-a|\). Explain your answer.

\item What is the basic idea behind representing a given function \(f(x)\) with a power series?

\item Give an example of a function that can be expressed with a power series based on geometric series manipulation.

\item How would you represent the function \(\frac{1}{1-x}\) as a power series?

\item Describe why it is sometimes necessary to truncate a power series when approximating a function in practice.

\item Explain the relationship between power series and functions that are infinitely differentiable near their center.

\item For a power series with finite radius of convergence \(R\), what can be said about convergence at the endpoints \(x=a\pm R\)?

\item How are the endpoints of the interval of convergence typically tested if the Ratio Test is inconclusive?

\item What implications does the convergence of a power series have for the differentiability and integrability of the represented function on its interval of convergence?

\end{enumerate}
\clearpage
\subsection*{6.2 Properties of Power Series}

\url{https://openstax.org/books/calculus-volume-2/pages/6-2-properties-of-power-series}

\begin{enumerate}

\item What is the general result about combining two power series
  \[
    \sum_{n=0}^{\infty}c_nx^n\quad\text{and}\quad\sum_{n=0}^{\infty}d_nx^n
  \]
  that converge on the same interval \(I\)? State the conditions for addition, scalar multiplication, and multiplication by a power of \(x\).

\item Suppose two power series converge to functions \(f(x)\) and \(g(x)\) on an interval \(I\). What function does the series
  \[
    \sum_{n=0}^{\infty}(c_n+d_n)x^n
  \]
  converge to on \(I\)?

\item For a power series \(\sum_{n=0}^{\infty}c_nx^n\) that converges to \(f(x)\) on an interval \(I\), what is the power series that converges to \(b\,x^m f(x)\) for a real number \(b\) and integer \(m\ge0\)?

\item If \(\sum_{n=0}^{\infty}c_nx^n\) converges to \(f(x)\) on an interval \(I\), what is the new series that represents \(f(b\,x^m)\) when \(|b\,x^m|\) is in the interval \(I\)?

\item What theorem guarantees that a power series can be differentiated and integrated term-by-term on its interval of convergence? What does this theorem say about the resulting series and their convergence?

\item Given a power series
  \[
    f(x)=\sum_{n=0}^{\infty}c_n(x-a)^n,
  \]
  write the term-by-term differentiated series for \(f'(x)\) in summation form.

\item Given the same \(f(x)\) above, write the term-by-term integrated series representing an antiderivative of \(f(x)\).

\item Does term-by-term differentiation or integration change the radius of convergence of a power series? Explain and describe what can happen at the endpoints.

\item Using the known geometric series for \(\frac{1}{1-x}\), what is the power series representation of \(\frac{1}{(1-x)^2}\) found by term-by-term differentiation? Specify its interval of convergence and endpoint behavior.

\item What is the interval of convergence and endpoint behavior for the series representation of \(\ln(1+x)\) found by term-by-term integration of \(\frac{1}{1+x}\)?

\item Why does term-by-term differentiation or integration *not* guarantee endpoint convergence even though it preserves radius of convergence? Describe a specific example.

\item Consider a power series that converges on \((-R,R)\). If \(x\) lies in this interval, explain whether the term-by-term derivative must converge at the same endpoints as the original series.

\end{enumerate}
\clearpage
\subsection*{6.3 Taylor and Maclaurin Series}

\url{https://openstax.org/books/calculus-volume-2/pages/6-3-taylor-and-maclaurin-series}

\begin{enumerate}

\item What is the general form of the Taylor series for a function \(f(x)\) centered at \(x=a\)? Write it using summation notation.

\item How does the Taylor series simplify when the center \(a=0\)? What name is given to this special case?

\item For a function \(f(x)\) with derivatives of all orders at \(x=a\), write the explicit expression for its \(n\)-th Taylor polynomial \(p_n(x)\) centered at \(a\).

\item Explain why the coefficients of a power series representation of \(f(x)\) at \(x=a\) must be given by derivatives of \(f\) evaluated at \(a\).

\item What condition must hold for a function’s Taylor series to equal the original function \(f(x)\) on an interval? Define the remainder \(R_n(x)\) in this context.

\item State the Uniqueness of Taylor Series theorem: if a power series converges to \(f(x)\) on an interval containing \(a\), what can be said about that series?

\item Define the remainder term \(R_n(x)\) for a Taylor polynomial approximation and write the formula for the Lagrange form of the remainder.

\item What does Taylor’s Theorem with remainder tell us about the error when approximating \(f(x)\) using its \(n\)-th Taylor polynomial?

\item Given a specific function \(f(x)\), describe the steps to find the Maclaurin polynomial of degree \(n\).

\item Find the Maclaurin series for \(e^x\) by writing out several derivatives at 0 and using the definition.

\item Find the Maclaurin series for \(\sin x\) and \(\cos x\). Explain the pattern in the derivatives that allows you to write the general term.

\item How can the ratio test be used to determine the radius of convergence for a Taylor or Maclaurin series?

\item For a given Taylor series, how would you check whether it actually converges to \(f(x)\) (not just converges in value)?

\item Why is it significant that Maclaurin polynomials are just Taylor polynomials centered at zero?

\item Suppose you are given a function \(f(x)\) with known derivatives at \(x=a\). Write a question asking you to compute the 2nd and 3rd Taylor polynomials explicitly and compare them to the function near \(a\).

\item Create a question that asks you to estimate the error in using the 3rd Taylor polynomial to approximate a function value at a given point using the remainder bound.

\item Why does the remainder \(R_n(x)\to 0\) as \(n\to\infty\) matter for Taylor series convergence?

\item Provide the Maclaurin series expansion for \(\ln(1+x)\) and specify its interval of convergence.

\item For a given nonzero center \(a\), how would you find the Taylor series of \(\ln(x)\) about \(x=a\) and determine its interval of convergence?

\item Write a practice question that asks for the Taylor series of a non-elementary function (e.g., \(e^{-x^2}\)) and discuss whether it converges to the function.

\item Formulate a question about how to bound the remainder \(R_n(x)\) using information about the maximum value of a derivative on an interval.

\item Write a question that asks you to use Taylor's theorem to prove that the Maclaurin series for \(e^x\) converges for all real \(x\).

\item Create a problem asking you to approximate a definite integral using a Taylor or Maclaurin series where the antiderivative is non-elementary.

\end{enumerate}
\clearpage
\subsection*{6.4 Working with Taylor Series}

\url{https://openstax.org/books/calculus-volume-2/pages/6-4-working-with-taylor-series}

\begin{enumerate}

\item What is the Taylor series of a function $f$ centered at $x=a$?

\item What is the Maclaurin series? How does it relate to the Taylor series?

\item Write the Taylor series of $f$ centered at $a$ using sigma notation.

\item What is the $n$th Taylor polynomial $T_n(x)$ of a function $f$ centered at $a$?

\item What is the relationship between the Taylor polynomial $T_n(x)$ and the full Taylor series?

\item Compute the Maclaurin series of $e^x$.

\item Compute the Maclaurin series of $\sin x$.

\item Compute the Maclaurin series of $\cos x$.

\item Compute the Maclaurin series of $\dfrac{1}{1-x}$ and state its interval of convergence.

\item How can you obtain the Maclaurin series for $\ln(1+x)$ from the geometric series?

\item How can you obtain the Maclaurin series for $\arctan x$?

\item Given a known power series, how can you:
  \begin{enumerate}
    \item Differentiate it term-by-term?
    \item Integrate it term-by-term?
  \end{enumerate}

\item What happens to the radius of convergence when you differentiate or integrate a power series?

\item How can you construct the Taylor series for $e^{2x}$ using the known series for $e^x$?

\item How can you construct the Taylor series for $\sin(3x)$?

\item How can you construct the Taylor series for $x e^x$?

\item How can you construct the Taylor series for $\dfrac{x}{1-x^2}$?

\item How do you find the Taylor series of a function centered at a point $a \neq 0$?

\item Find the Taylor series of $\ln x$ centered at $x=1$.

\item What substitution allows you to rewrite $\ln x$ in terms of $\ln(1+u)$?

\item How can you approximate a function using its Taylor polynomial?

\item What is Taylor's theorem?

\item What is the remainder term $R_n(x)$ in Taylor's theorem?

\item State the Lagrange form of the remainder.

\item How can you use the remainder term to bound the error of a Taylor approximation?

\item Suppose $|f^{(n+1)}(x)| \le M$ on an interval containing $a$ and $x$. What inequality bounds $|R_n(x)|$?

\item How do you determine how large $n$ must be to guarantee a given approximation error?

\item Use a Taylor polynomial to approximate $e^{0.1}$. How can you estimate the error?

\item Use a Taylor polynomial to approximate $\sin(0.2)$. How can you estimate the error?

\item What does it mean for a function to be equal to its Taylor series?

\item Give an example of a function whose Taylor series converges but does not equal the function.

\item Why does the Taylor series of $e^x$ converge for all $x$?

\item Why does the geometric series converge only for $|x|<1$?

\item How does the factorial in the denominator affect convergence of Taylor series?

\item When approximating near $x=a$, why is it better to center the Taylor polynomial at $a$ rather than at $0$?

\item If you need a very accurate approximation near $x=2$, which center should you choose and why?

\item How does increasing $n$ affect:
  \begin{enumerate}
    \item The accuracy of approximation?
    \item The computational cost?
  \end{enumerate}

\item What is the difference between a Taylor polynomial and a Taylor series?

\item How does Taylor series provide a bridge between polynomials and general smooth functions?

\end{enumerate}
\clearpage


\section*{Cauchy Formula for Repeated Integration}

\subsection*{Basic understanding}

\begin{enumerate}
\item What problem does the Cauchy formula for repeated integration solve?

\item What is meant by the "$n$-th repeated integral" of a function?

\item Write the definition of the $n$-fold repeated integral of a function $f$ with the base point $a$:
  \[
    f^{(-n)}(x) = ?
  \]

\item Under what conditions on $f$ does the formula hold?

\item What is the main result of the Cauchy formula for repeated integration? State it explicitly.
\end{enumerate}

\subsection*{Formula and structure}

\begin{enumerate}
\item Rewrite the repeated integral as a single integral:
  \[
    f^{(-n)}(x) = \frac{1}{(n-1)!} \int_a^x (x - t)^{n-1} f(t)\,dt.
  \]

\item What is the role of the factor $\frac{1}{(n-1)!}$?

\item Why does the kernel $(x - t)^{n-1}$ appear in the formula?

\item What happens in the special case $n = 1$? Verify that the formula reduces correctly.

\item Compute explicitly $f^{(-2)}(x)$ using both:
  \begin{itemize}
    \item the definition (nested integrals),
    \item the Cauchy formula,
  \end{itemize}
  and verify that they agree.
\end{enumerate}

\subsection*{Proof understanding}

\begin{enumerate}
\item What proof technique is used to prove the formula?

\item What is the base case of the induction?

\item Show that for $n=1$:
  \[
    f^{(-1)}(x) = \int_a^x f(t)\,dt.
  \]

\item What assumption is made in the induction step?

\item What must be proven in the induction step?

\item Where is the Leibniz integral rule used in the proof?

\item Compute:
  \[
    \frac{d}{dx} \left[ \frac{1}{n!} \int_a^x (x - t)^n f(t)\,dt \right]
  \]
  and explain why it gives the desired recursive relation.

\item How does the induction step transform an $(n+1)$-fold integral into the desired single-integral form?

\item Why does the induction argument prove the formula for all $n \in \mathbb{N}$?
\end{enumerate}

\subsection*{Conceptual understanding}

\begin{enumerate}
\item Why can repeated integration be "compressed" into a single integral?

\item How is this formula related to convolution?

\item Interpret the kernel $(x - t)^{n-1}$ geometrically or intuitively.

\item How does this formula simplify practical computations?

\item In what situations is this formula especially useful?
\end{enumerate}

\subsection*{Generalizations}

\begin{enumerate}
\item How can the formula be extended to non-integer orders of integration?

\item What replaces the factorial $(n-1)!$ in the generalized version?

\item State the fractional integral version:
  \[
    (J^\alpha f)(x) = ?
  \]

\item What condition must $\alpha$ satisfy?

\item What is the name of this generalization?

\item How does the formula relate to fractional derivatives (when $\alpha < 0$)?
\end{enumerate}

\subsection*{Connections and further ideas}

\begin{enumerate}
\item How is this formula connected to fractional calculus?

\item What is a differintegral?

\item How can repeated integration be used to define differentiation of non-integer order?

\item How is this formula generalized to higher dimensions?

\item What is the Riesz potential?
\end{enumerate}

\subsection*{Practice problems}

\begin{enumerate}
\item Compute:
  \[
    \int_0^x \int_0^{t_1} t_2 \, dt_2 \, dt_1
  \]
  using the Cauchy formula.

\item Compute the $n$-fold integral of $f(t) = 1$.

\item Compute the $n$-fold integral of $f(t) = t^k$.

\item Show that:
  \[
    (J^n f)(x) = (f * g_n)(x)
  \]
  for an appropriate kernel $g_n$.

\item Verify the formula numerically for a simple function (e.g. $f(t)=e^t$).
\end{enumerate}
\clearpage
\input{misc/change-of-variables}\clearpage
\section*{Differentiation Under the Integral Sign}

\begin{enumerate}

\item What is the statement of Leibniz's rule for differentiating under the integral sign with constant limits?

\item What is the general form of Leibniz's rule when the limits of integration depend on the parameter?

\item What conditions must be satisfied to justify differentiation under the integral sign (continuity, uniform convergence, dominated convergence, etc.)?

\item What role does continuity of the integrand and its partial derivative play in applying differentiation under the integral sign?

\item How does the Dominated Convergence Theorem justify differentiation under the integral sign?

\item When can uniform convergence be used to justify differentiation under the integral sign?

\item What is the difference between pointwise convergence and uniform convergence in this context?

\item How do you compute $\frac{d}{d\alpha} \int_a^b f(x,\alpha),dx$?

\item How do you compute $\frac{d}{d\alpha} \int_{a(\alpha)}^{b(\alpha)} f(x,\alpha),dx$?

\item What additional terms appear when the limits of integration depend on the parameter?

\item How do you apply differentiation under the integral sign to evaluate difficult integrals?

\item What is the idea behind Feynman's technique for evaluating integrals using parameters?

\item How can introducing a parameter simplify the evaluation of an integral?

\item Give an example where differentiating under the integral sign transforms an integral into a simpler one.

\item How do you recover the original integral after differentiating with respect to a parameter?

\item What initial or boundary conditions are needed after integrating back with respect to the parameter?

\item How do you handle constants of integration when using this method?

\item When is it valid to interchange differentiation and integration?

\item What are common pitfalls when applying differentiation under the integral sign?

\item How does this technique apply to improper integrals?

\item What extra care is needed when the interval of integration is infinite?

\item How do you justify differentiation under the integral sign for $\int_0^\infty f(x,\alpha),dx$?

\item Can differentiation under the integral sign be applied multiple times? Under what conditions?

\item What is the relationship between differentiation under the integral sign and parameter-dependent integrals?

\item How does this method connect to solving differential equations?

\item How can this technique be used in probability theory (e.g., moment generating functions)?

\item How does differentiation under the integral sign relate to Fourier transforms or Laplace transforms?

\item What are some classic integrals that are evaluated using this method?

\item How would you construct your own parameterized integral to apply this technique?

\item How do you verify that your final answer is correct after applying the method?

\end{enumerate}
\clearpage
\section*{Euler's Method}

\begin{enumerate}

\item What problem does Euler's method aim to solve? State the general form of an initial value problem.

\item What is the geometric interpretation of Euler's method in terms of tangent lines?

\item Starting from the differential equation $y'(x) = f(x,y)$, derive the Euler update formula.

\item Write the recursive formula for Euler's method, defining all variables involved.

\item What role does the step size $h$ play in Euler's method?

\item Given an initial condition $y(x_0) = y_0$, how do you compute $y_1$ using Euler's method?

\item How do you compute $y_{n+1}$ from $y_n$ in Euler's method?

\item What is meant by a numerical approximation to a solution of an ODE?

\item What is the local truncation error in Euler's method?

\item What is the global truncation error in Euler's method?

\item What is the order of accuracy of Euler's method?

\item How does decreasing the step size $h$ affect the accuracy of Euler's method?

\item What is the trade-off between accuracy and computational cost in Euler's method?

\item Apply Euler's method with step size $h = 0.1$ to approximate the solution of $y' = y$, $y(0) = 1$ at $x = 0.1$.

\item Apply Euler's method for two steps to approximate the solution of $y' = x + y$, $y(0) = 1$, with $h = 0.5$.

\item What types of errors can accumulate in Euler's method over many steps?

\item What is meant by stability in the context of numerical methods for ODEs?

\item Why can Euler's method become unstable for certain differential equations?

\item Give an example of a differential equation where Euler's method performs poorly.

\item What is the difference between explicit Euler and implicit Euler methods?

\item Write the update formula for the implicit Euler method.

\item What advantage does the implicit Euler method have over the explicit version?

\item What is a stiff differential equation, and why is Euler's method problematic for such equations?

\item How can you modify Euler's method to improve accuracy (e.g., midpoint or improved Euler methods)?

\item What is the idea behind the Heun method (improved Euler method)?

\item Compare Euler's method with higher-order methods like Runge--Kutta methods.

\item Why is Euler's method rarely used in practice for high-precision computations?

\item How can you estimate the error in Euler's method during computation?

\item What happens if the function $f(x,y)$ is not smooth? How does this affect Euler's method?

\item In what situations might Euler's method still be useful despite its limitations?

\item How would you implement Euler's method algorithmically?

\item Write pseudocode for Euler's method.

\item How does Euler's method relate to the Taylor series expansion of the solution?

\item What term(s) of the Taylor series does Euler's method include or neglect?

\item Can Euler's method be used for systems of differential equations? How?

\item How does Euler's method behave when applied backward in time (negative step size)?

\item What are common pitfalls when using Euler's method in practice?

\item How would you visualize the steps of Euler's method on a slope field?

\item What is the difference between consistency, stability, and convergence in the context of Euler's method?

\item State the relationship between consistency, stability, and convergence (Lax equivalence theorem).

\end{enumerate}
\clearpage
\section*{The Gamma Function}

\begin{enumerate}

\item What is the definition of the Gamma function $\Gamma(x)$ for $x > 0$?

\item For which values of $x$ is the integral definition of $\Gamma(x)$ valid?

\item Show that $\Gamma(1) = 1$.

\item Prove the functional equation $\Gamma(x+1) = x\Gamma(x)$.

\item How does the Gamma function extend the factorial function? State the relationship between $\Gamma(n)$ and $(n-1)!$ for $n \in \mathbb{N}$.

\item Compute $\Gamma(2)$, $\Gamma(3)$, and $\Gamma(4)$ using the functional equation.

\item What is the value of $\Gamma\left(\tfrac{1}{2}\right)$?

\item How can the integral $\int_0^\infty e^{-t^2},dt$ be used to compute $\Gamma\left(\tfrac{1}{2}\right)$?

\item What substitution transforms the Gaussian integral into the Gamma function integral?

\item What is the reflection formula for the Gamma function?

\item Use the reflection formula to compute $\Gamma\left(\tfrac{1}{2}\right)$.

\item What is the duplication formula (Legendre's formula) for the Gamma function?

\item What is the behavior of $\Gamma(x)$ near $x = 0$?

\item Does the Gamma function have any poles? If so, where are they located?

\item Is the Gamma function defined for negative non-integer values? What happens at negative integers?

\item What is the logarithmic derivative of the Gamma function called?

\item Define the digamma function $\psi(x)$.

\item What is the integral representation of $\Gamma(x)$ involving $e^{-t} t^{x-1}$?

\item What is an alternative integral representation of $\Gamma(x)$ using a limit (Euler's limit form)?

\item What is Stirling's approximation for $\Gamma(x)$ for large $x$?

\item How can Stirling's formula be used to approximate $n!$?

\item What is the Beta function $B(x,y)$?

\item How is the Beta function related to the Gamma function?

\item Show that $B(x,y) = \dfrac{\Gamma(x)\Gamma(y)}{\Gamma(x+y)}$.

\item What is the integral definition of the Beta function?

\item How can you derive the Beta-Gamma relationship using a change of variables?

\item What is the asymptotic growth rate of $\Gamma(x)$ as $x \to \infty$?

\item Is the Gamma function convex or concave on $(0,\infty)$?

\item What is the logarithmic convexity (log-convexity) property of $\Gamma(x)$?

\item State Bohr–Mollerup theorem and explain its significance.

\item How does the Gamma function behave under scaling transformations?

\item What is the value of $\Gamma\left(\tfrac{3}{2}\right)$?

\item Compute $\Gamma\left(\tfrac{5}{2}\right)$ using recursion.

\item Express $\Gamma\left(n+\tfrac{1}{2}\right)$ in terms of factorials and $\sqrt{\pi}$.

\item What is the relation between $\Gamma(x)$ and improper integrals in probability theory?

\item How does the Gamma function appear in the definition of the Gamma distribution?

\item What is the probability density function of the Gamma distribution?

\item What role does $\Gamma(x)$ play in normalizing probability distributions?

\item How can $\Gamma(x)$ be analytically continued to complex numbers?

\item Is $\Gamma(x)$ an entire function? Why or why not?

\item Where are the zeros of the Gamma function located?

\item What is the Weierstrass product representation of $\Gamma(x)$?

\item How does $\Gamma(x)$ relate to complex analysis and contour integrals?

\item What is Euler's constant $\gamma$ and how does it relate to $\Gamma(x)$?

\item What is the series expansion of $\log \Gamma(x)$ near $x=1$?

\item How is the Gamma function used in evaluating definite integrals?

\item Give an example of an integral that evaluates to a Gamma function.

\item How does the substitution $t = -\ln u$ relate the Gamma function to integrals over $(0,1)$?

\item What is the connection between $\Gamma(x)$ and factorial moments?

\item Why is the Gamma function important in physics and engineering?

\end{enumerate}
\clearpage
\section*{Hessian Matrix}

\subsection*{Basic Definitions}
\begin{enumerate}
\item What is the Hessian matrix of a scalar-valued function $f:\mathbb{R}^n \to \mathbb{R}$?
\item Write the general form of the Hessian matrix for a function $f(x_1, x_2, \dots, x_n)$.
\item What are the entries of the Hessian matrix in terms of partial derivatives?
\item For $f(x,y)$, write the Hessian explicitly.
\end{enumerate}

\subsection*{Computation}
\begin{enumerate}
\item Compute the Hessian of $f(x,y) = x^2 + y^2$.
\item Compute the Hessian of $f(x,y) = x^2 y + y^3$.
\item Compute the Hessian of $f(x,y,z) = xyz$.
\item Compute the Hessian of $f(x,y) = e^{x^2 + y^2}$.
\end{enumerate}

\subsection*{Interpretation}
\begin{enumerate}
\item What information does the Hessian matrix encode about a function?
\item How is the Hessian related to the second-order Taylor expansion?
\item What does the Hessian tell you about local curvature?
\end{enumerate}

\subsection*{Critical Points and Optimization}
\begin{enumerate}
\item What is a critical point of a multivariable function?
\item State the second derivative test using the Hessian for functions of two variables.
\item What are the conditions on the Hessian for:
\item a local minimum?
\item a local maximum?
\item a saddle point?
\item What does it mean for the Hessian to be positive definite?
\item What does it mean for the Hessian to be negative definite?
\item What happens if the Hessian is indefinite?
\item What happens if the determinant of the Hessian is zero?
\end{enumerate}

\subsection*{Matrix Properties}
\begin{enumerate}
\item When is the Hessian symmetric?
\item State the conditions under which mixed partial derivatives are equal.
\item Why is symmetry of the Hessian important?
\end{enumerate}

\subsection*{Eigenvalues and Geometry}
\begin{enumerate}
\item How are eigenvalues of the Hessian related to curvature?
\item What does it mean if all eigenvalues are positive?
\item What does it mean if eigenvalues have mixed signs?
\item How do eigenvectors of the Hessian relate to principal directions?
\end{enumerate}

\subsection*{Advanced Connections}
\begin{enumerate}
\item How does the Hessian relate to convexity of a function?
\item State the condition for a function to be convex using the Hessian.
\item What is the Hessian in the context of optimization algorithms (e.g., Newton's method)?
\item How is the Hessian used in quadratic approximation?
\item What is the bordered Hessian and when is it used?
\end{enumerate}

\subsection*{Conceptual Checks}
\begin{enumerate}
\item Why does the first derivative test fail for multivariable functions without second-order information?
\item Can a function have a zero gradient but not be an extremum? Explain using the Hessian.
\item How does the Hessian generalize the second derivative from single-variable calculus?
\end{enumerate}
\clearpage
\input{misc/IEEE-754}\clearpage
\section*{Derivative of the Inverse Function}

\begin{enumerate}

\item What is the definition of the inverse of a function $f$? Under what condition does an inverse exist?

\item What does it mean for a function to be one-to-one (injective), and why is this property necessary for the existence of an inverse?

\item State the relationship between a function and its inverse using composition.

\item If $y = f(x)$, how do you express $x$ in terms of $f^{-1}$?

\item What is the geometric relationship between the graphs of $f$ and $f^{-1}$?

\item How can you determine whether a function has an inverse using the horizontal line test?

\item State the formula for the derivative of an inverse function: \[ (f^{-1})'(x) = ? \]

\item Derive the formula for $\left(f^{-1}\right)'(x)$ starting from the identity $f(f^{-1}(x)) = x$.

\item If $f(a) = b$, express $\left(f^{-1}\right)'(b)$ in terms of $f'(a)$.

\item Under what conditions on $f$ does the derivative of the inverse function exist?

\item Why must $f'(a) \neq 0$ for the inverse function to be differentiable at $b = f(a)$?

\item Compute $\left(f^{-1}\right)'(x)$ when $f(x) = x^3 + x$.

\item Let $f(x) = \sin x$ restricted to $\left[-\frac{\pi}{2}, \frac{\pi}{2}\right]$. What is $f^{-1}(x)$ and its derivative?

\item Find the derivative of $\arctan(x)$ using the inverse function derivative formula.

\item Show that the derivative of $\ln x$ can be derived from the inverse of $e^x$.

\item If $f'(x)$ is known, describe the steps to compute $\left(f^{-1}\right)'(x)$ at a specific point.

\item What happens to $\left(f^{-1}\right)'(x)$ if $f'(x)$ is very large? What is the geometric interpretation?

\item What happens to $\left(f^{-1}\right)'(x)$ if $f'(x)$ is very small (but nonzero)?

\item Explain why the slopes of inverse functions at corresponding points are reciprocals.

\item If $f$ is decreasing and invertible, what can you say about the sign of $\left(f^{-1}\right)'(x)$?

\item Let $f(x) = x^5$. Compute $\left(f^{-1}\right)'(x)$ explicitly.

\item If $f(x) = \tan x$ on $\left(-\frac{\pi}{2}, \frac{\pi}{2}\right)$, compute the derivative of its inverse.

\item Verify that $\frac{d}{dx}(\arcsin x) = \frac{1}{\sqrt{1 - x^2}}$ using the inverse function rule.

\item How does implicit differentiation relate to finding the derivative of an inverse function?

\item If $f$ is differentiable and strictly monotonic on an interval, what can be said about the differentiability of $f^{-1}$?

\item Give an example of a function that is invertible but whose inverse is not differentiable at some point.

\item What is the domain of $f^{-1}$ in terms of the range of $f$?

\item What is the range of $f^{-1}$ in terms of the domain of $f$?

\item How can you use the inverse function derivative formula to avoid explicitly finding the inverse?

\item Let $f(2) = 5$ and $f'(2) = 3$. Find $\left(f^{-1}\right)'(5)$.

\end{enumerate}
\clearpage
\input{misc/jacobian}\clearpage
\section*{Newton's Method}

\begin{enumerate}

\item What is the goal of a root-finding algorithm?

\item State the Newton (Newton--Raphson) method iteration formula for finding a root of a function $f(x)$.

\item How is the Newton iteration formula derived from the tangent line approximation?

\item Write the equation of the tangent line to $f(x)$ at a point $x_n$.

\item How do you obtain the next iterate $x_{n+1}$ from the tangent line?

\item What assumptions about $f(x)$ are required for Newton's method to work?

\item What is meant by an initial guess $x_0$, and why is it important?

\item Define what it means for Newton's method to converge.

\item What is quadratic convergence?

\item Under what conditions does Newton's method exhibit quadratic convergence?

\item What happens if $f'(x_n) = 0$ during the iteration?

\item Why can Newton's method fail to converge for some initial guesses?

\item Give an example of a function and initial guess where Newton's method diverges.

\item What is meant by the basin of attraction of a root?

\item How does the choice of initial guess affect which root Newton's method converges to?

\item What is a multiple root, and how does it affect convergence of Newton's method?

\item How can Newton's method be modified to handle multiple roots?

\item What stopping criteria can be used to terminate Newton's method?

\item Compare stopping criteria based on $|x_{n+1} - x_n|$ and $|f(x_n)|$.

\item What is the computational cost of one Newton iteration?

\item How does Newton's method compare to the bisection method in terms of speed and reliability?

\item Why is Newton's method considered a local method?

\item What is the geometric interpretation of Newton's method?

\item How can Newton's method be extended to systems of nonlinear equations?

\item Write the Newton iteration for a system of equations using the Jacobian matrix.

\item What role does the Jacobian matrix play in multidimensional Newton's method?

\item What issues arise when the Jacobian is singular or nearly singular?

\item What is a damped (or modified) Newton method?

\item Why might damping improve convergence?

\item How can Newton's method be used to find extrema of a function?

\item How is Newton's method related to solving $f'(x) = 0$?

\item What is the difference between Newton's method for optimization and for root finding?

\item What is the order of convergence of Newton's method compared to secant and bisection methods?

\item What is the secant method, and how does it approximate Newton's method?

\item When might the secant method be preferred over Newton's method?

\item What is the effect of numerical errors in evaluating $f(x)$ or $f'(x)$?

\item How can automatic differentiation help in implementing Newton's method?

\item What are practical safeguards used in implementations of Newton's method?

\item What is line search in the context of Newton's method?

\item How can Newton's method behave near inflection points?

\item What happens if the function is not differentiable at the root?

\item How can Newton's method be visualized graphically?

\item Why can Newton's method oscillate between points instead of converging?

\item How can you detect divergence in practice?

\item What are hybrid methods that combine Newton's method with bracketing methods?

\item Why are hybrid methods often preferred in practice?

\item What is the role of scaling in Newton's method for systems?

\item How does conditioning of the problem affect convergence?

\item What is meant by the error $e_n = x_n - r$, and how does it evolve in Newton's method?

\item Derive the error recurrence relation for Newton's method near a simple root.

\item Why does Newton's method converge faster near the root than far from it?

\end{enumerate}
\clearpage
\section*{Roots of Unity}

\begin{enumerate}

\item What is meant by an $n$th root of unity?

\item Write the equation whose solutions are the $n$th roots of unity.

\item How many distinct $n$th roots of unity are there? Why?

\item Express the $n$th roots of unity in exponential form using Euler's formula.

\item Write the general formula for the $k$th $n$th root of unity.

\item What is the principal (or primitive) $n$th root of unity?

\item How can all $n$th roots of unity be generated from a single primitive root?

\item What is the difference between a root of unity and a primitive root of unity?

\item State the condition under which a root of unity is primitive.

\item How many primitive $n$th roots of unity are there? Which function counts them?

\item What is Euler's totient function $\varphi(n)$ and how does it relate to roots of unity?

\item Represent the $n$th roots of unity geometrically in the complex plane.

\item What is the magnitude and argument of each $n$th root of unity?

\item How are the $n$th roots of unity distributed on the complex unit circle?

\item What is the angle between consecutive $n$th roots of unity?

\item Compute the sum of all $n$th roots of unity. What is the result?

\item Why does the sum of all $n$th roots of unity equal zero?

\item Compute the product of all $n$th roots of unity.

\item What symmetry properties do the $n$th roots of unity exhibit?

\item Show that the $n$th roots of unity form a group under multiplication.

\item What type of group is formed by the $n$th roots of unity?

\item What is the identity element in this group?

\item What is the inverse of a given $n$th root of unity?

\item Show that the set of $n$th roots of unity is cyclic.

\item What is a generator of the group of $n$th roots of unity?

\item For which values of $k$ is $\zeta^k$ a primitive root, where $\zeta$ is a primitive $n$th root?

\item Solve the equation $z^n = 1$ explicitly for small values such as $n=2,3,4$.

\item What are the cube roots of unity? Write them explicitly.

\item What are the fourth roots of unity? Write them explicitly.

\item How can roots of unity be used to factor the polynomial $x^n - 1$?

\item Write the factorization of $x^n - 1$ in terms of its complex roots.

\item What are cyclotomic polynomials and how are they related to roots of unity?

\item Define the $n$th cyclotomic polynomial $\Phi_n(x)$.

\item How does $x^n - 1$ factor into cyclotomic polynomials?

\item What is the relationship between primitive roots of unity and cyclotomic polynomials?

\item How can roots of unity be used to evaluate sums such as $\sum_{k=0}^{n-1} \zeta^{k}$?

\item Evaluate $\sum_{k=0}^{n-1} \zeta^{mk}$ for an integer $m$.

\item Under what condition does $\sum_{k=0}^{n-1} \zeta^{mk} = 0$?

\item What happens when $m$ is a multiple of $n$ in the above sum?

\item How are roots of unity used in discrete Fourier transforms (DFT)?

\item What is the connection between roots of unity and periodicity?

\item How can roots of unity simplify computations involving periodic sums?

\item Show that if $\zeta$ is an $n$th root of unity, then $\overline{\zeta} = \zeta^{-1}$.

\item What is the complex conjugate of a root of unity geometrically?

\item How do roots of unity relate to rotations in the complex plane?

\item What is the minimal polynomial of a primitive $n$th root of unity?

\item How does the degree of $\Phi_n(x)$ relate to $\varphi(n)$?

\item In what fields do primitive $n$th roots of unity lie?

\item What is the significance of roots of unity in number theory?

\item How are roots of unity used in solving polynomial equations?

\end{enumerate}
\clearpage

\section*{Complex Numbers as Matrices / Representation Theory}

\url{https://www.youtube.com/watch?v=hsveVFoIJPM}

\begin{enumerate}

\item What does it mean to represent an abstract algebraic object (like a group element or number) as a matrix?

\item What is the general idea of a \textbf{representation} in the context of linear algebra and group theory?

\item How can a complex number $a + bi$ be represented as a $2 \times 2$ real matrix?

\item Write explicitly the matrix corresponding to the complex number $a + bi$.

\item Show how matrix addition corresponds to addition of complex numbers under this representation.

\item Show how matrix multiplication corresponds to multiplication of complex numbers under this representation.

\item Why is this matrix representation of complex numbers structure-preserving?

\item What algebraic structure is preserved when mapping complex numbers to matrices?

\item What property must a mapping satisfy to be considered a homomorphism?

\item Verify that the mapping from complex numbers to matrices is a homomorphism.

\item What is the identity element in the complex numbers, and what matrix represents it?

\item What matrix corresponds to the imaginary unit $i$?

\item Compute the square of the matrix representing $i$ and interpret the result.

\item How does this matrix representation encode the fact that $i^2 = -1$?

\item What geometric transformation in $\mathbb{R}^2$ corresponds to multiplication by $i$?

\item How does multiplication by a general complex number $a + bi$ act geometrically on the plane?

\item How is rotation represented in terms of matrices?

\item How does scaling appear in the matrix representation of a complex number?

\item Explain how complex multiplication combines rotation and scaling.

\item What is the determinant of the matrix corresponding to $a + bi$?

\item What is the geometric meaning of this determinant?

\item What is the trace of the matrix representation of $a + bi$?

\item How does the modulus $|a+bi|$ relate to the determinant of the matrix?

\item What condition on $a$ and $b$ makes the matrix invertible?

\item What is the matrix corresponding to the inverse of a complex number?

\item Show that the inverse matrix corresponds to the multiplicative inverse of the complex number.

\item How does this representation help visualize complex numbers as linear transformations?

\item What is the connection between this representation and rotations in the plane?

\item How can this idea be generalized to represent other algebraic structures using matrices?

\item What is a group representation in general?

\item Why are matrices particularly useful for representing abstract algebraic objects?

\item How does this example motivate the study of representation theory?

\item What advantages do matrix representations provide for computation?

\item How can eigenvalues of these matrices be interpreted in terms of complex numbers?

\item What happens when you diagonalize the matrix corresponding to a complex number?

\item How does this representation connect linear algebra with complex analysis?

\item In what sense are complex numbers ``the same'' as a subset of $2 \times 2$ real matrices?

\item What is the dimension of the vector space in which these matrices act?

\item How does this perspective change your understanding of multiplication by complex numbers?

\item Can every linear transformation of $\mathbb{R}^2$ be represented by a complex number? Why or why not?

\item What distinguishes matrices that correspond to complex numbers from arbitrary $2 \times 2$ matrices?

\item How would you test whether a given $2 \times 2$ matrix corresponds to a complex number?

\item What deeper insight does this example give about the relationship between algebra and geometry?

\end{enumerate}
\clearpage
\section*{Geometric Meaning of the Third Derivative}

\url{https://www.youtube.com/watch?v=SovllrJUQ64}

\begin{enumerate}

\item What is the geometric meaning of the first derivative $f'(x)$ in terms of the graph of $f(x)$?

\item How can you interpret the second derivative $f''(x)$ geometrically?

\item What does it mean for $f''(x) > 0$ in terms of the shape of the graph?

\item What does it mean for $f''(x) < 0$?

\item Define an inflection point in terms of concavity.

\item What condition on $f''(x)$ is necessary for an inflection point?

\item Why is $f''(x)=0$ not sufficient to guarantee an inflection point?

\item What additional condition ensures that a point where $f''(x)=0$ is actually an inflection point?

\item How can sign changes of $f''(x)$ be used to detect inflection points?

\item What is the geometric meaning of the third derivative $f'''(x)$?

\item How can $f'''(x)$ be interpreted as a rate of change of another geometric quantity?

\item If $f'''(x) > 0$, what does this say about how concavity is changing?

\item If $f'''(x) < 0$, what does this imply about the change in concavity?

\item How does $f'''(x)$ describe the “bending behavior” of a curve beyond concavity?

\item Explain how the third derivative relates to the steepness of the tangent slope.

\item How does the graph of $f'(x)$ help visualize $f''(x)$?

\item How does the graph of $f''(x)$ help visualize $f'''(x)$?

\item If $f''(x)$ has a local maximum or minimum, what can you say about $f'''(x)$ at that point?

\item What is the relationship between $f'''(x)=0$ and extrema of $f''(x)$?

\item Can $f'''(x)=0$ correspond to an inflection point of $f(x)$? Explain.

\item How would you distinguish between:
  \begin{itemize}
    \item an inflection point of $f(x)$
    \item an extremum of $f''(x)$
  \end{itemize}

\item Describe how successive derivatives correspond to successive “rates of change” geometrically.

\item Give a physical interpretation of $f'(x)$, $f''(x)$, and $f'''(x)$ in motion (position, velocity, acceleration, ...).

\item What is the physical meaning of the third derivative in kinematics (often called jerk)?

\item How does the sign of the third derivative affect motion in a physical system?

\item Construct an example function where $f''(x)=0$ but there is no inflection point.

\item Construct an example where $f'''(x)=0$ but $f''(x)$ does not change sign.

\item Given a graph, how would you visually estimate where $f'''(x)$ is positive or negative?

\item How can you use higher derivatives to understand increasingly subtle features of a curve?

\item Why is the third derivative rarely discussed compared to the first and second in basic calculus?

\item In what types of problems does understanding the third derivative become important?

\end{enumerate}
\clearpage
\section*{Second Derivative Test and Hessian}

\url{https://www.youtube.com/watch?v=z0bSdV8DA4M}

\subsection*{Core Concept: What is the second derivative test really doing?}

\begin{itemize}
  \item What does the second derivative test attempt to determine about a function at a critical point?
  \item Why is checking only the first derivative insufficient for classifying critical points?
  \item In one variable, what does the sign of $f''(x)$ tell us geometrically?
  \item How can you interpret $f''(x)$ in terms of curvature?
\end{itemize}

\subsection*{From 1D to Multivariable Functions}

\begin{itemize}
  \item What is the analogue of the second derivative for multivariable functions?
  \item What is the Hessian matrix, and how is it constructed?
  \item Why can’t we just look at ``the second derivative'' in multiple dimensions?
  \item What does the Hessian represent geometrically?
\end{itemize}

\subsection*{Quadratic Approximation}

\begin{itemize}
  \item What is the second-order Taylor expansion of a function near a critical point?
  \item Why does the linear term vanish at a critical point?
  \item What role does the quadratic term play in determining local behavior?
  \item How does the function locally resemble a quadratic form?
\end{itemize}

\subsection*{Quadratic Forms and Geometry}

\begin{itemize}
  \item What is a quadratic form?
  \item How can the expression $\mathbf{x}^T H \mathbf{x}$ describe local curvature?
  \item What does it mean for a quadratic form to be:
      \begin{itemize}
        \item Positive definite?
        \item Negative definite?
        \item Indefinite?
      \end{itemize}
\end{itemize}

\subsection*{Classification of Critical Points}

\begin{itemize}
  \item How does the definiteness of the Hessian classify a critical point?
  \item What condition corresponds to a local minimum?
  \item What condition corresponds to a local maximum?
  \item What condition corresponds to a saddle point?
  \item What happens if the Hessian is degenerate (determinant zero)?
\end{itemize}

\subsection*{Eigenvalues Interpretation}

\begin{itemize}
  \item Why are eigenvalues the key to understanding the Hessian?
  \item How do eigenvalues relate to curvature along different directions?
  \item What does it mean if all eigenvalues are positive?
  \item What does it mean if eigenvalues have mixed signs?
  \item How do eigenvectors relate to principal directions of curvature?
\end{itemize}

\subsection*{Geometric Intuition}

\begin{itemize}
  \item Why is a saddle point associated with ``curving up in some directions and down in others''?
  \item How can you visualize curvature along different directions through a point?
  \item Why is the second derivative test really a statement about curvature in all directions?
\end{itemize}

\subsection*{Connection to Linear Algebra}

\begin{itemize}
  \item Why can any symmetric matrix be diagonalized?
  \item How does diagonalizing the Hessian simplify the quadratic form?
  \item What does the diagonal form reveal about the function locally?
\end{itemize}

\subsection*{Deeper Understanding}

\begin{itemize}
  \item Why is the second derivative test fundamentally about approximating the function by a quadratic surface?
  \item In what sense is the test coordinate-independent?
  \item How does this perspective generalize beyond $\mathbb{R}^2$ or $\mathbb{R}^3$?
\end{itemize}

\subsection*{Edge Cases and Limitations}

\begin{itemize}
  \item Why does the test fail when eigenvalues are zero?
  \item What additional analysis is needed in degenerate cases?
  \item Can higher-order terms change the classification?
\end{itemize}
\clearpage


\end{document}
