\subsection*{5.1 Sequences}

\url{https://openstax.org/books/calculus-volume-2/pages/5-1-sequences}

\begin{enumerate}

\item What is an infinite sequence? How is it typically denoted?

\item How can a sequence be interpreted as a function? What is its domain?

\item What is the index of a sequence, and what role does it play?

\item What is an explicit formula for a sequence? Give an example.

\item What is a recursive definition of a sequence? How does it differ from an explicit formula?

\item Given a recursive sequence, how can you compute its first few terms?

\item What does it mean to graph a sequence? How does it differ from graphing a function?

\item What is the limit of a sequence?

\item What does it mean for a sequence ${a_n}$ to converge to a number $L$?

\item Write the formal $\epsilon$–$N$ definition of convergence of a sequence.

\item What does it mean for a sequence to diverge?

\item What are some different ways a sequence can diverge?

\item What does it mean for a sequence to diverge to infinity? To negative infinity?

\item Why does changing a finite number of terms of a sequence not affect its convergence?

\item How can limits of functions be used to determine limits of sequences?

\item State the theorem relating $\lim_{x\to\infty} f(x)$ and $\lim_{n\to\infty} a_n$ when $a_n = f(n)$.

\item Evaluate $\lim_{n\to\infty} \frac{1}{n}$ using a function-based argument.

\item What happens to $r^n$ as $n \to \infty$ when:
  \begin{itemize}
    \item $0 < r < 1$?
    \item $r = 1$?
    \item $r > 1$?
  \end{itemize}

\item How can limit laws be applied to sequences?

\item How do you find the limit of a sequence that is a sum or product of simpler sequences?

\item What is a bounded sequence?

\item What does it mean for a sequence to be increasing? Decreasing?

\item What is a monotonic (monotone) sequence?

\item State the Monotone Convergence Theorem.

\item Why is the Monotone Convergence Theorem useful?

\item Give an example of a sequence that is bounded and increasing.

\item Give an example of a sequence that is unbounded.

\item Give an example of a sequence that oscillates and does not converge.

\item How can you determine convergence from the behavior of the terms as $n \to \infty$?

\item What is the difference between intuition (“approaches a value”) and the formal definition of convergence?

\item How can sequences be used to model real-world processes or iterative procedures?

\item How are sequences related to infinite series?

\end{enumerate}
