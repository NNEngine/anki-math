\subsection*{6.3 Taylor and Maclaurin Series}

\url{https://openstax.org/books/calculus-volume-2/pages/6-3-taylor-and-maclaurin-series}

\begin{enumerate}

\item What is the general form of the Taylor series for a function \(f(x)\) centered at \(x=a\)? Write it using summation notation.

\item How does the Taylor series simplify when the center \(a=0\)? What name is given to this special case?

\item For a function \(f(x)\) with derivatives of all orders at \(x=a\), write the explicit expression for its \(n\)-th Taylor polynomial \(p_n(x)\) centered at \(a\).

\item Explain why the coefficients of a power series representation of \(f(x)\) at \(x=a\) must be given by derivatives of \(f\) evaluated at \(a\).

\item What condition must hold for a function’s Taylor series to equal the original function \(f(x)\) on an interval? Define the remainder \(R_n(x)\) in this context.

\item State the Uniqueness of Taylor Series theorem: if a power series converges to \(f(x)\) on an interval containing \(a\), what can be said about that series?

\item Define the remainder term \(R_n(x)\) for a Taylor polynomial approximation and write the formula for the Lagrange form of the remainder.

\item What does Taylor’s Theorem with remainder tell us about the error when approximating \(f(x)\) using its \(n\)-th Taylor polynomial?

\item Given a specific function \(f(x)\), describe the steps to find the Maclaurin polynomial of degree \(n\).

\item Find the Maclaurin series for \(e^x\) by writing out several derivatives at 0 and using the definition.

\item Find the Maclaurin series for \(\sin x\) and \(\cos x\). Explain the pattern in the derivatives that allows you to write the general term.

\item How can the ratio test be used to determine the radius of convergence for a Taylor or Maclaurin series?

\item For a given Taylor series, how would you check whether it actually converges to \(f(x)\) (not just converges in value)?

\item Why is it significant that Maclaurin polynomials are just Taylor polynomials centered at zero?

\item Suppose you are given a function \(f(x)\) with known derivatives at \(x=a\). Write a question asking you to compute the 2nd and 3rd Taylor polynomials explicitly and compare them to the function near \(a\).

\item Create a question that asks you to estimate the error in using the 3rd Taylor polynomial to approximate a function value at a given point using the remainder bound.

\item Why does the remainder \(R_n(x)\to 0\) as \(n\to\infty\) matter for Taylor series convergence?

\item Provide the Maclaurin series expansion for \(\ln(1+x)\) and specify its interval of convergence.

\item For a given nonzero center \(a\), how would you find the Taylor series of \(\ln(x)\) about \(x=a\) and determine its interval of convergence?

\item Write a practice question that asks for the Taylor series of a non-elementary function (e.g., \(e^{-x^2}\)) and discuss whether it converges to the function.

\item Formulate a question about how to bound the remainder \(R_n(x)\) using information about the maximum value of a derivative on an interval.

\item Write a question that asks you to use Taylor's theorem to prove that the Maclaurin series for \(e^x\) converges for all real \(x\).

\item Create a problem asking you to approximate a definite integral using a Taylor or Maclaurin series where the antiderivative is non-elementary.

\end{enumerate}
