\subsection*{4.3 Separable Equations}

\url{https://openstax.org/books/calculus-volume-2/pages/4-3-separable-equations}

\begin{enumerate}

\item What is a separable differential equation?

\item In what form must a differential equation be written to be considered separable?

\item Write the general form:
  \[
    \frac{dy}{dx} = f(x)g(y).
  \]

\item Why is such an equation called “separable”?

\item What does it mean to separate variables in a differential equation?

\item How do you rewrite a separable differential equation in differential form?

\item Write the separated form:
  \[
    \frac{dy}{g(y)} = f(x)\,dx.
  \]

\item What is the main idea behind the method of separation of variables?

\item What are the steps of the separation of variables method?

\item Why must you check for values where $g(y)=0$ before separating variables?

\item What do solutions of $g(y)=0$ represent?

\item Why can constant solutions be lost during the separation process?

\item After separating variables, what is the next step?

\item Why do we integrate both sides of the equation?

\item What form does the equation take after integration?

\item Why is an arbitrary constant $C$ introduced after integration?

\item When solving, why might it not be possible to express $y$ explicitly as a function of $x$?

\item What is an implicit solution?

\item Give an example of an implicit solution arising from separation of variables.

\item How do you apply an initial condition to a separable differential equation?

\item What is an initial-value problem in this context?

\item After integrating, how do you solve for the constant using an initial condition?

\item Solve the general structure:
  \[
    \int \frac{1}{g(y)}\,dy = \int f(x)\,dx.
  \]

\item For the equation $y' = (x^2 - 4)(3y+2)$, how do you separate variables?

\item What constant solution arises from $3y+2=0$?

\item After separation, what integral must be computed:
  \[
    \int \frac{dy}{3y+2}?
  \]

\item What substitution is useful for integrating expressions like $\frac{1}{3y+2}$?

\item Why is substitution often needed when integrating the $y$-side?

\item How do you interpret the final solution after integrating both sides?

\item For the equation $y' = (2x+3)(y^2 - 4)$, how do you separate variables?

\item What constant solutions arise from $y^2 - 4 = 0$?

\item Why do these constant solutions need to be included separately?

\item What integration technique is typically used for expressions like $\frac{1}{y^2 - 4}$?

\item How does partial fraction decomposition appear in separable equations?

\item Why are integration techniques from earlier calculus essential here?

\item What is an autonomous differential equation?

\item How can you recognize an autonomous equation from its form?

\item Why is an equation of the form $y' = g(y)$ separable?

\item What is the general solution to an equation of the form $y' = f(x)$?

\item What is the general solution to an equation of the form $y' = g(y)$?

\item How does separation of variables generalize these simpler cases?

\item In what situations is separation of variables especially useful?

\item Why are separable equations common in physics and engineering?

\item What are the limitations of the separation of variables method?

\item Can every differential equation be solved using separation of variables?

\item How can you recognize when a differential equation is not separable?

\item What is the overall strategy when solving separable differential equations?

\end{enumerate}
