\subsection*{4.2 Direction Fields and Numerical Methods}

\url{https://openstax.org/books/calculus-volume-2/pages/4-2-direction-fields-and-numerical-methods}

\begin{enumerate}

\item What is a direction field (slope field) for a differential equation?

\item For what type of differential equations are direction fields typically used?

\item What is the general form of a first-order differential equation used in direction fields?

\item What does each small line segment in a direction field represent?

\item How is the slope of a line segment at a point $(x,y)$ determined?

\item How do you construct a direction field for a differential equation $y' = f(x,y)$?

\item Why can a direction field be drawn without solving the differential equation explicitly?

\item What qualitative information can a direction field provide about solutions?

\item What is a solution curve in the context of a direction field?

\item How can you sketch a solution curve using a direction field?

\item Why do solution curves follow the line segments in a direction field?

\item What is an initial condition in the context of direction fields?

\item How do you use an initial point $(x_0,y_0)$ to sketch a particular solution?

\item Why do solution curves corresponding to different initial conditions typically not intersect?

\item What is an equilibrium solution?

\item How can equilibrium solutions be identified from a differential equation?

\item How do equilibrium solutions appear in a direction field?

\item What does it mean for an equilibrium solution to be stable?

\item What does it mean for an equilibrium solution to be unstable?

\item How can you determine stability visually from a direction field?

\item What is the purpose of numerical methods in differential equations?

\item Why are numerical methods needed when solving differential equations?

\item What is Euler’s Method?

\item What type of problems is Euler’s Method used to approximate?

\item What is the idea behind Euler’s Method in terms of tangent line approximation?

\item What is the step size $h$ in Euler’s Method?

\item How does the choice of step size affect the accuracy of Euler’s Method?

\item What is the iterative formula for Euler’s Method?

\item Starting from $(x_n, y_n)$, how do you compute $(x_{n+1}, y_{n+1})$?

\item Write the update rule:
  \[
    y_{n+1} = y_n + h f(x_n, y_n).
  \]

\item How is $x_{n+1}$ computed from $x_n$?

\item Apply Euler’s Method to approximate the solution of
  \[
    y' = f(x,y), \quad y(x_0)=y_0
  \]
  for one step.

\item Apply Euler’s Method for two steps and express the result explicitly.

\item What geometric idea explains why Euler’s Method works?

\item Why does Euler’s Method accumulate error over multiple steps?

\item What is the difference between the exact solution and a numerical approximation?

\item How can decreasing the step size $h$ improve the approximation?

\item What is the trade-off when choosing a very small step size?

\item How is Euler’s Method related to linear approximation (tangent lines)?

\item In what sense does Euler’s Method “follow” the direction field?

\item How would you approximate a solution curve using only a direction field (without formulas)?

\item What are the limitations of direction fields?

\item What are the limitations of Euler’s Method?

\item When would you prefer a graphical method over an analytical solution?

\item When would you prefer a numerical method over solving analytically?

\item How can direction fields and Euler’s Method be used together?

\end{enumerate}
