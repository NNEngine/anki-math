\subsection*{4.5 First-order Linear Equations}

\url{https://openstax.org/books/calculus-volume-2/pages/4-5-first-order-linear-equations}

\begin{enumerate}

\item What is a first-order linear differential equation?

\item What is the general (standard) form of a first-order linear differential equation?

\item Write the standard form:
  \[
    y' + p(x)y = q(x).
  \]

\item How can any first-order linear differential equation be rewritten into standard form?

\item What roles do the functions $p(x)$ and $q(x)$ play?

\item Why is the equation called ``linear''?

\item What distinguishes a linear differential equation from a separable one?

\item Can every linear differential equation be solved by separation of variables? Why or why not?

\item What is an integrating factor?

\item Why is an integrating factor useful when solving linear differential equations?

\item What property do we want the integrating factor to create?

\item What is the formula for the integrating factor $\mu(x)$?

\item Write:
  \[
    \mu(x) = e^{\int p(x)\,dx}.
  \]

\item How is the integrating factor derived?

\item What differential equation does $\mu(x)$ satisfy?

\item Write:
  \[
    \mu'(x) = \mu(x)p(x).
  \]

\item After multiplying by $\mu(x)$, what form does the equation take?

\item Why does the left-hand side become a derivative of a product?

\item Write the identity:
  \[
    \frac{d}{dx}(\mu(x)y) = \mu(x)y' + \mu(x)p(x)y.
  \]

\item After multiplying by $\mu(x)$, what equation do we obtain?

\item Write:
  \[
    \frac{d}{dx}(\mu(x)y) = \mu(x)q(x).
  \]

\item What is the next step after obtaining this form?

\item Why can both sides now be integrated easily?

\item After integrating, what form does the solution take?

\item Write the general solution:
  \[
    \mu(x)y = \int \mu(x)q(x)\,dx + C.
  \]

\item How do you solve for $y(x)$ after integration?

\item What is the full step-by-step strategy for solving first-order linear equations?

\item List the five steps of the method.

\item Why is it important to first rewrite the equation in standard form?

\item What happens if the coefficient of $y'$ is not 1?

\item How do you handle an equation of the form $a(x)y' + b(x)y = c(x)$?

\item Why is dividing by $a(x)$ necessary?

\item What is an initial-value problem in this context?

\item How do you apply an initial condition to determine the constant $C$?

\item Solve symbolically:
  \[
    y' + p(x)y = 0.
  \]

\item What is the general solution to the homogeneous equation?

\item How does the homogeneous solution relate to the general solution?

\item What is the difference between homogeneous and nonhomogeneous equations?

\item What happens when $q(x)=0$?

\item What happens when $q(x)\neq 0$?

\item Why does the solution involve an integral of $\mu(x)q(x)$?

\item What integration techniques are often required in solving these equations?

\item How does the integrating factor method simplify the problem conceptually?

\item What is the geometric interpretation of multiplying by $\mu(x)$?

\item Why does the method always work for first-order linear equations?

\item What are common mistakes when applying the integrating factor method?

\item How can you verify that a solution is correct?

\item In what applications do first-order linear equations appear?

\item How are they used in modeling motion with air resistance?

\item How are they used in electrical circuits?

\item Why are first-order linear equations important in applied mathematics?

\end{enumerate}
