\subsection*{7.3 Polar Coordinates}

\url{https://openstax.org/books/calculus-volume-2/pages/7-3-polar-coordinates}

\begin{enumerate}

\item What is the polar coordinate system, and how does it differ from the Cartesian coordinate system?

\item What do the coordinates $(r,\theta)$ represent in the polar coordinate system?

\item What is the geometric meaning of the radial coordinate $r$?

\item What is the geometric meaning of the angle $\theta$?

\item What are the pole and the polar axis?

\item How are positive and negative angles interpreted in polar coordinates?

\item What happens when the radial coordinate $r$ is negative?

\item Why does a single point in the plane have infinitely many polar representations?

\item How can the same point be represented using $(r,\theta)$ and $(r,\theta + 2\pi)$?

\item How can a point be represented using a negative radius and a shifted angle?

\item What are the conversion formulas from polar to rectangular coordinates?

\item What are the conversion formulas from rectangular to polar coordinates?

\item How can you graph a polar equation $r = f(\theta)$?

\item Why are polar coordinates especially useful for describing circular or radial patterns?

\item What types of curves are naturally expressed in polar coordinates (e.g., circles, spirals)?

\item What does it mean for a polar graph to have symmetry about the pole?

\item What does it mean for a polar graph to have symmetry about the line $\theta = \frac{\pi}{2}$?

\item How can symmetry help simplify graphing polar equations?

\item What is a polar equation, and how does it define a curve?

\item How do you interpret the equation $r = c$ geometrically?

\item How do you interpret the equation $\theta = c$ geometrically?

\end{enumerate}
