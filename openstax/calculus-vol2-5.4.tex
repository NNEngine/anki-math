\subsection*{5.4 Comparison Tests}

\url{https://openstax.org/books/calculus-volume-2/pages/5-4-comparison-tests}

\begin{enumerate}

\item What is the main idea behind comparison tests for infinite series?

\item For what type of series (in terms of sign of terms) are comparison tests typically used?

\item Why are geometric series and $p$-series commonly used in comparison tests?

\item State the Direct Comparison Test for series $\sum a_n$ and $\sum b_n$ with positive terms.

\item If $0 \le a_n \le b_n$ and $\sum b_n$ converges, what can be concluded about $\sum a_n$?

\item If $0 \le b_n \le a_n$ and $\sum b_n$ diverges, what can be concluded about $\sum a_n$?

\item In which cases is the Direct Comparison Test inconclusive?

\item How do you choose an appropriate comparison series $b_n$ for a given series $a_n$?

\item Why is it useful to compare a series to a $p$-series of the form $\sum \frac{1}{n^p}$?

\item What is the behavior of the $p$-series $\sum \frac{1}{n^p}$ for different values of $p$?

\item What is the behavior of a geometric series $\sum ar^n$ depending on the value of $r$?

\item State the Limit Comparison Test for two series $\sum a_n$ and $\sum b_n$ with positive terms.

\item What does it mean if
  \[
    \lim_{n \to \infty} \frac{a_n}{b_n} = c,
  \]
  where $0 < c < \infty$?

\item What can be concluded if the limit in the Limit Comparison Test is $0$?

\item What can be concluded if the limit in the Limit Comparison Test is $\infty$?

\item When is the Limit Comparison Test more useful than the Direct Comparison Test?

\item Why does the Limit Comparison Test work even when inequalities between $a_n$ and $b_n$ are hard to establish?

\item What are common functions or expressions you simplify when applying the Limit Comparison Test?

\item How do you simplify rational expressions in $n$ when applying comparison tests?

\item How do logarithmic terms (e.g., $\ln n$) affect comparison with $p$-series?

\item How do roots (e.g., $\sqrt{n}$) affect comparison with $p$-series?

\item Give an example of a series that can be compared to $\sum \frac{1}{n^2}$ and explain why.

\item Give an example of a series that can be compared to $\sum \frac{1}{n}$ and explain why.

\item What is the general strategy for proving convergence using comparison tests?

\item What is the general strategy for proving divergence using comparison tests?

\item Why must terms $a_n$ and $b_n$ be positive for these tests?

\item Can comparison tests determine the exact sum of a series? Why or why not?

\item How do comparison tests relate to the Integral Test conceptually?

\item What role do asymptotic behaviors of functions play in the Limit Comparison Test?

\item How can you justify that two sequences have the same “growth rate” for comparison?

\item When comparing $\frac{1}{n^2+1}$ to $\frac{1}{n^2}$, why does the comparison work?

\end{enumerate}
