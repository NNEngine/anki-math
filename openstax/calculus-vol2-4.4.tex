\subsection*{4.4 The Logistic Equation}

\url{https://openstax.org/books/calculus-volume-2/pages/4-4-the-logistic-equation}

\begin{enumerate}

\item What real-world limitation motivates the logistic model of population growth?

\item Why is exponential growth not realistic for large populations?

\item What is the carrying capacity of a population?

\item What does the symbol $K$ represent in the logistic equation?

\item What does the symbol $r$ represent in the logistic equation?

\item Write the logistic differential equation:
  \[
    \frac{dP}{dt} = rP\left(1 - \frac{P}{K}\right).
  \]

\item Why is the logistic equation nonlinear?

\item What happens to $\frac{dP}{dt}$ when $P$ is very small?

\item How does the logistic equation behave when $P \ll K$?

\item How does the logistic model approximate exponential growth for small populations?

\item What happens to $\frac{dP}{dt}$ when $P = K$?

\item What does it mean physically when $P = K$?

\item What happens to $\frac{dP}{dt}$ when $P > K$?

\item Why does the population decrease when $P > K$?

\item What are the equilibrium solutions of the logistic equation?

\item Solve:
  \[
    rP\left(1 - \frac{P}{K}\right) = 0.
  \]

\item Why are $P=0$ and $P=K$ equilibrium solutions?

\item What is meant by stability of an equilibrium solution?

\item Is $P=0$ stable, unstable, or semi-stable? Explain.

\item Is $P=K$ stable, unstable, or semi-stable? Explain.

\item What is a phase line?

\item How do you construct a phase line for the logistic equation?

\item What does the phase line indicate about the sign of $\frac{dP}{dt}$?

\item For which values of $P$ is the population increasing?

\item For which values of $P$ is the population decreasing?

\item How does the phase line describe long-term behavior of solutions?

\item What happens to $P(t)$ as $t \to \infty$ when $0 < P_0 < K$?

\item What happens to $P(t)$ as $t \to \infty$ when $P_0 > K$?

\item Why does the solution approach $K$ but never reach it exactly?

\item What is the general strategy to solve the logistic differential equation?

\item Why is the logistic equation separable?

\item Rewrite the logistic equation in separated form.

\item What algebraic manipulation is needed before integrating?

\item Why is partial fraction decomposition used in solving the logistic equation?

\item What is the general solution of the logistic differential equation?

\item Write the solution:
  \[
    P(t)=\frac{P_0 K e^{rt}}{(K-P_0)+P_0 e^{rt}}.
  \]

\item What is the role of the initial condition $P(0)=P_0$?

\item How does the solution behave as $t \to \infty$?

\item How does the solution behave as $t \to -\infty$?

\item What is the shape of the graph of a logistic function?

\item Why is the logistic curve called sigmoidal (S-shaped)?

\item What is a point of inflection?

\item Why does the logistic solution have a point of inflection?

\item At what population value does the inflection point occur?

\item Why is the growth rate maximal at $P = \frac{K}{2}$?

\item How can you find the inflection point using the second derivative?

\item What happens to concavity before and after the inflection point?

\item How does the logistic model improve on exponential growth models?

\item In what types of real-world systems is the logistic equation used?

\item What assumptions underlie the logistic model?

\item What are the limitations of the logistic equation?

\end{enumerate}
