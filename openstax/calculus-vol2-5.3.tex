\subsection*{5.3 The Divergence and Integral Tests}

\url{https://openstax.org/books/calculus-volume-2/pages/5-3-the-divergence-and-integral-tests}

\begin{enumerate}

\item Why is it often difficult to determine convergence of a series by directly evaluating partial sums?

\item What is the divergence test (nth-term test)?

\item State the divergence test formally.

\item Why does $\sum_{n=1}^{\infty} a_n$ converging imply that $a_n \to 0$?

\item If $\lim_{n\to\infty} a_n \neq 0$, what can you conclude about the series $\sum a_n$?

\item If $\lim_{n\to\infty} a_n = 0$, what can you conclude about the series $\sum a_n$?

\item Why is the divergence test only useful for proving divergence and not convergence?

\item Give an example of a series where $a_n \to 0$ but the series diverges.

\item How do you apply the divergence test in practice to a given series?

\item What kinds of limits of $a_n$ guarantee divergence (nonzero limit, limit does not exist)?

\item What is the idea behind comparing a series to an improper integral?

\item What is the integral test?

\item What conditions must a function $f(x)$ satisfy to apply the integral test (positivity, continuity, monotonicity)?

\item How are the series $\sum_{n=1}^{\infty} a_n$ and the integral $\int_1^\infty f(x),dx$ related in the integral test?

\item State the integral test formally.

\item If the improper integral $\int_1^\infty f(x),dx$ converges, what can you conclude about the series?

\item If the improper integral diverges, what can you conclude about the series?

\item How can the integral test be used to prove that the harmonic series diverges?

\item What is a $p$-series? Write its general form.

\item For what values of $p$ does the $p$-series $\sum_{n=1}^{\infty} \frac{1}{n^p}$ converge?

\item For what values of $p$ does the $p$-series diverge?

\item How can the integral test be used to establish the convergence properties of $p$-series?

\item What is the remainder $R_N$ of a series after $N$ terms?

\item How can the integral test be used to estimate the remainder of a series?

\item State the inequality involving the remainder $R_N$ and integrals.

\item How do upper and lower bounds for $R_N$ arise from the integral test?

\item Why is it useful to estimate the error when approximating a series by partial sums?

\item How can you choose $N$ to ensure the error is less than a desired tolerance?

\item What is the geometric interpretation of the integral test using areas under curves?

\item How does monotonicity of $f(x)$ play a role in bounding the series with integrals?

\item In what situations is the integral test particularly useful compared to other tests?

\end{enumerate}
