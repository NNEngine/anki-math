\subsection*{2.4 Arc Length of a Curve and Surface Area}

\url{https://openstax.org/books/calculus-volume-2/pages/2-4-arc-length-of-a-curve-and-surface-area}

\begin{enumerate}

\item How can the length of a curve be approximated using line segments?

\item How does partitioning an interval lead to a Riemann sum for arc length?

\item What is the formula for the arc length of a curve $y=f(x)$ on $[a,b]$?

\item What assumptions must be satisfied by $f(x)$ for the arc length formula to apply?

\item Why does the arc length formula involve $\sqrt{1+[f'(x)]^2}$?

\item How is the Pythagorean theorem used in deriving the arc length formula?

\item How do you compute arc length when the curve is given as $x=g(y)$?

\item How does the derivation of arc length for $x=g(y)$ differ from that for $y=f(x)$?

\item What role do smoothness and differentiability play in arc length problems?

\item How can you decide whether to integrate with respect to $x$ or $y$?

\item What are common substitutions used when evaluating arc length integrals?

\item How can symmetry simplify arc length computations?

\item What is a surface of revolution?

\item How can a surface of revolution be approximated using frustums of cones?

\item What is a frustum, and why is it useful in this context?

\item How does approximating a surface with frustums lead to a Riemann sum?

\item How is the surface area formula derived from arc length ideas?

\item What is the formula for the surface area when rotating $y=f(x)$ about the $x$-axis?

\item What is the formula for the surface area when rotating $x=f(y)$ about the $y$-axis?

\item Why does the surface area formula include a factor of $2\pi$?

\item What geometric quantity does the factor $2\pi f(x)$ or $2\pi x$ represent?

\item How does arc length appear inside the surface area integral?

\item How do you modify the surface area formula when the curve is given as $x=g(y)$?

\item What similarities exist between arc length and surface area formulas?

\end{enumerate}
