\section*{Second Derivative Test and Hessian}

\url{https://www.youtube.com/watch?v=z0bSdV8DA4M}

\subsection*{Core Concept: What is the second derivative test really doing?}

\begin{itemize}
  \item What does the second derivative test attempt to determine about a function at a critical point?
  \item Why is checking only the first derivative insufficient for classifying critical points?
  \item In one variable, what does the sign of $f''(x)$ tell us geometrically?
  \item How can you interpret $f''(x)$ in terms of curvature?
\end{itemize}

\subsection*{From 1D to Multivariable Functions}

\begin{itemize}
  \item What is the analogue of the second derivative for multivariable functions?
  \item What is the Hessian matrix, and how is it constructed?
  \item Why can’t we just look at ``the second derivative'' in multiple dimensions?
  \item What does the Hessian represent geometrically?
\end{itemize}

\subsection*{Quadratic Approximation}

\begin{itemize}
  \item What is the second-order Taylor expansion of a function near a critical point?
  \item Why does the linear term vanish at a critical point?
  \item What role does the quadratic term play in determining local behavior?
  \item How does the function locally resemble a quadratic form?
\end{itemize}

\subsection*{Quadratic Forms and Geometry}

\begin{itemize}
  \item What is a quadratic form?
  \item How can the expression $\mathbf{x}^T H \mathbf{x}$ describe local curvature?
  \item What does it mean for a quadratic form to be:
      \begin{itemize}
        \item Positive definite?
        \item Negative definite?
        \item Indefinite?
      \end{itemize}
\end{itemize}

\subsection*{Classification of Critical Points}

\begin{itemize}
  \item How does the definiteness of the Hessian classify a critical point?
  \item What condition corresponds to a local minimum?
  \item What condition corresponds to a local maximum?
  \item What condition corresponds to a saddle point?
  \item What happens if the Hessian is degenerate (determinant zero)?
\end{itemize}

\subsection*{Eigenvalues Interpretation}

\begin{itemize}
  \item Why are eigenvalues the key to understanding the Hessian?
  \item How do eigenvalues relate to curvature along different directions?
  \item What does it mean if all eigenvalues are positive?
  \item What does it mean if eigenvalues have mixed signs?
  \item How do eigenvectors relate to principal directions of curvature?
\end{itemize}

\subsection*{Geometric Intuition}

\begin{itemize}
  \item Why is a saddle point associated with ``curving up in some directions and down in others''?
  \item How can you visualize curvature along different directions through a point?
  \item Why is the second derivative test really a statement about curvature in all directions?
\end{itemize}

\subsection*{Connection to Linear Algebra}

\begin{itemize}
  \item Why can any symmetric matrix be diagonalized?
  \item How does diagonalizing the Hessian simplify the quadratic form?
  \item What does the diagonal form reveal about the function locally?
\end{itemize}

\subsection*{Deeper Understanding}

\begin{itemize}
  \item Why is the second derivative test fundamentally about approximating the function by a quadratic surface?
  \item In what sense is the test coordinate-independent?
  \item How does this perspective generalize beyond $\mathbb{R}^2$ or $\mathbb{R}^3$?
\end{itemize}

\subsection*{Edge Cases and Limitations}

\begin{itemize}
  \item Why does the test fail when eigenvalues are zero?
  \item What additional analysis is needed in degenerate cases?
  \item Can higher-order terms change the classification?
\end{itemize}
