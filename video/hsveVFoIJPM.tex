\section*{Complex Numbers as Matrices / Representation Theory}

\url{https://www.youtube.com/watch?v=hsveVFoIJPM}

\begin{enumerate}

\item What does it mean to represent an abstract algebraic object (like a group element or number) as a matrix?

\item What is the general idea of a \textbf{representation} in the context of linear algebra and group theory?

\item How can a complex number $a + bi$ be represented as a $2 \times 2$ real matrix?

\item Write explicitly the matrix corresponding to the complex number $a + bi$.

\item Show how matrix addition corresponds to addition of complex numbers under this representation.

\item Show how matrix multiplication corresponds to multiplication of complex numbers under this representation.

\item Why is this matrix representation of complex numbers structure-preserving?

\item What algebraic structure is preserved when mapping complex numbers to matrices?

\item What property must a mapping satisfy to be considered a homomorphism?

\item Verify that the mapping from complex numbers to matrices is a homomorphism.

\item What is the identity element in the complex numbers, and what matrix represents it?

\item What matrix corresponds to the imaginary unit $i$?

\item Compute the square of the matrix representing $i$ and interpret the result.

\item How does this matrix representation encode the fact that $i^2 = -1$?

\item What geometric transformation in $\mathbb{R}^2$ corresponds to multiplication by $i$?

\item How does multiplication by a general complex number $a + bi$ act geometrically on the plane?

\item How is rotation represented in terms of matrices?

\item How does scaling appear in the matrix representation of a complex number?

\item Explain how complex multiplication combines rotation and scaling.

\item What is the determinant of the matrix corresponding to $a + bi$?

\item What is the geometric meaning of this determinant?

\item What is the trace of the matrix representation of $a + bi$?

\item How does the modulus $|a+bi|$ relate to the determinant of the matrix?

\item What condition on $a$ and $b$ makes the matrix invertible?

\item What is the matrix corresponding to the inverse of a complex number?

\item Show that the inverse matrix corresponds to the multiplicative inverse of the complex number.

\item How does this representation help visualize complex numbers as linear transformations?

\item What is the connection between this representation and rotations in the plane?

\item How can this idea be generalized to represent other algebraic structures using matrices?

\item What is a group representation in general?

\item Why are matrices particularly useful for representing abstract algebraic objects?

\item How does this example motivate the study of representation theory?

\item What advantages do matrix representations provide for computation?

\item How can eigenvalues of these matrices be interpreted in terms of complex numbers?

\item What happens when you diagonalize the matrix corresponding to a complex number?

\item How does this representation connect linear algebra with complex analysis?

\item In what sense are complex numbers ``the same'' as a subset of $2 \times 2$ real matrices?

\item What is the dimension of the vector space in which these matrices act?

\item How does this perspective change your understanding of multiplication by complex numbers?

\item Can every linear transformation of $\mathbb{R}^2$ be represented by a complex number? Why or why not?

\item What distinguishes matrices that correspond to complex numbers from arbitrary $2 \times 2$ matrices?

\item How would you test whether a given $2 \times 2$ matrix corresponds to a complex number?

\item What deeper insight does this example give about the relationship between algebra and geometry?

\end{enumerate}
